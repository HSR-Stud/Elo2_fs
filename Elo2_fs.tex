%%This is a very basic article template.
%%There is just one section and two subsections.
\documentclass{article}

%definitions for Formelsammlung

\usepackage[left=1.5cm,right=1.5cm,top=2.5cm,bottom=2cm,landscape]{geometry} 
\usepackage{multicol}
\usepackage[ngerman]{babel}
\usepackage{tabularx}
\usepackage{mathpazo}
\usepackage{mathtools}
\usepackage{amsmath}  
\usepackage{setspace} 
\usepackage{commath}
\usepackage[utf8]{inputenc}
%\usepackage[ansinew]{inputenc}  
\usepackage[T1]{fontenc}
\usepackage{lmodern} 
\usepackage{hyperref}
\usepackage{bigints}
\usepackage{array}
\usepackage{xcolor}
\usepackage{layouts}
\usepackage{siunitx}
\usepackage{wrapfig}
\usepackage{multirow,bigstrut}
\usepackage{trfsigns}
\usepackage{amssymb} 
\usepackage{fancyhdr}
\usepackage{datetime}
\usepackage{pgfplots}
\usepgfplotslibrary{fillbetween}
\usepackage{listings}
\usepackage{mathrsfs}
\usepackage{booktabs}
\usepackage{wrapfig}
%\usepackage[europeanvoltages, europeancurrents, europeanresistors, americaninductors]{circuitikz}


\DeclareMathOperator\arctanh{arctanh}
\DeclareMathOperator\arsinh{arsinh} 
\DeclareMathOperator\arcosh{arcosh}
\DeclareMathOperator\artanh{artanh}
\DeclareMathOperator\arcoth{arcoth} 
\DeclareMathOperator\sinc{sinc} 
\DeclareMathOperator\sgn{sgn} 
\DeclareMathOperator\LPF{LPF} 
\DeclareMathOperator\Q{Q} 
\DeclareMathOperator\erf{erf} 


%colorCodes
\definecolor{listinggray}{gray}{0.9}
\definecolor{lbcolor}{rgb}{0.95,0.95,0.95}
\definecolor{lightGray}{gray}{0.1}

\definecolor{cOrange}{HTML}{996633}
\definecolor{clOrange}{HTML}{DBB48D}
\definecolor{cBlue}{HTML}{336699}
\definecolor{clBlue}{HTML}{A0BCD8}
\definecolor{cGreen}{HTML}{339966}
\definecolor{clGreen}{HTML}{94D4B4}
\definecolor{cRed}{HTML}{993333}
\definecolor{clRed}{HTML}{D0B0B0}
\definecolor{cGray}{gray}{0.4}
\definecolor{clGray}{gray}{0.96}


\setlength{\parindent}{0pt}
%\DeclareMathOperator\arctanh{arccot}
\newcolumntype{L}[1]{>{\raggedright\let\newline\\\arraybackslash\hspace{0pt}}m{#1}}
\newcolumntype{C}[1]{>{\centering\let\newline\\\arraybackslash\hspace{0pt}}m{#1}}
\newcolumntype{R}[1]{>{\raggedleft\let\newline\\\arraybackslash\hspace{0pt}}m{#1}}
\newcolumntype{Y}{>{\centering\arraybackslash}X}
\newcolumntype{Z}{>{\raggedleft\arraybackslash}X}
\newcommand{\fmm}{\displaystyle} 
\newcommand{\cn}[1]{\underline{#1}} 
\newcommand{\hlaplace}{\quad\laplace\quad}
\newcommand{\hLaplace}{\quad\Laplace\quad}
\newcommand{\infint}{\int_{-\infty}^{+\infty}}
\newcommand{\infiint}{\iint_{-\infty}^{+\infty}}
\newcommand{\limint}{\lim_{T\rightarrow \infty} \frac{1}{T} \int_{-T/2}^{T/2}}
\newcommand{\bedeq}{\mathrel{\stackrel{\makebox[0pt]{\mbox{\normalfont\tiny WSS}}}{=\joinrel=}}}

\renewenvironment{description}{\color{cGray}}{}
\newenvironment{definition}{\color{cGray}}{}
\newcommand{\cdef}[1]{\begin{definition}#1\end{definition}}


\newcommand{\vLaplace}[1][]{\mbox{\setlength{\unitlength}{0.1em}%
        \begin{picture}(10,20)%
          \put(3,2){\circle{4}}%
          \put(3,4){\line(0,1){12}}%
          \put(3,18){\circle*{4}}%
          \put(10,7){#1}
        \end{picture}%
       }%
 }%

\newcommand{\vlaplace}[1][]{\mbox{\setlength{\unitlength}{0.1em}%
        \begin{picture}(10,20)%
          \put(3,2){\circle*{4}}%
          \put(3,4){\line(0,1){12}}%
          \put(3,18){\circle{4}}%
          \put(10,7){#1}
        \end{picture}%
       }%
 }%                    
 
 
 
\renewcommand{\arraystretch}{1.5}

\newenvironment{mtabular}[1] {
  \renewcommand{\arraystretch}{2}
  
  \begin{tabular}{#1}
}  
{
  \end{tabular}
  
  \renewcommand{\arraystretch}{1.5}
}

\newenvironment{dtabular} {
  \begin{tabular}{>{\begin{definition}}l<{\end{definition}} >{\begin{definition}}l<{\end{definition}}}
}  
{
  \end{tabular}
}

\newenvironment{ddtabular} {
  \begin{tabular}{>{\begin{definition}}l<{\end{definition}} >{\begin{definition}}l<{\end{definition}} >{\begin{definition}}l<{\end{definition}} >{\begin{definition}}l<{\end{definition}}}
}  
{
  \end{tabular}
}


%configure tikz
%arrow heads
\tikzset{>=latex}

%system description
\usetikzlibrary{shapes,arrows,arrows.meta}
\tikzstyle{block} = [draw, rectangle, minimum height=3em, minimum width=4em]
\tikzstyle{input} = [coordinate]
\tikzstyle{output} = [coordinate]
\tikzstyle{pinstyle} = [pin edge={to-,thin,black}]
\tikzstyle{sum} = [draw, circle, node distance=1em, minimum height=1.5em]

\newcommand{\bipolLength}{/tikz/circuitikz/bipoles/length}

%Signal Fluss Diagramm

\usetikzlibrary{decorations.markings}
\newif\iflabrev


\newenvironment{sfd}[1]
{\begin{tikzpicture} [
  scale=#1,
  transform shape,
  label revd/.is if=labrev,
  %label revd/.default=true,
  amark/.style={
    decoration={             
      markings,  
      mark=at position {0.5} with { 
        \arrow{latex},
        \iflabrev \node[above] {##1};\else \node[below] {##1};\fi
      }
    },
    font=\footnotesize, 
    postaction={decorate}
  },
  amarkback/.style={
    decoration={             
      markings,  
      mark=at position {0.5} with { 
        \arrow{latex},
        \iflabrev \node[below, rotate=180] {##1};\else \node[above, rotate=180] {##1};\fi
      }
    },
    font=\footnotesize, 
    postaction={decorate}
  },
  terminal/.style 2 args={draw,circle, fill=black, scale=0.5, inner sep=2pt,label={##1:##2}},
  ]
}
{\end{tikzpicture}}


%USAGE:

%place a node:
%\node[terminal={below}{$V_{in }$}] (vin ) at (0,0) {};
%\node[terminal={below}{$V_{out}$}] (vout) at (3,0) {};

%Draw connector
%\draw[amark=$a$] (vin) to (vout);
%\draw[amarkback=$b$] (vout) to [bend right=50] (vin);




%lstlisting

\lstset{
  backgroundcolor=\color{lbcolor},
  tabsize=2,    
% rulecolor=,
  language=[GNU]C++,
  basicstyle=\scriptsize,
  upquote=true,
  aboveskip={1.5\baselineskip},
  columns=fixed,
  showstringspaces=false,
  extendedchars=false,
  breaklines=true,
  prebreak = \raisebox{0ex}[0ex][0ex]{\ensuremath{\hookleftarrow}},
  frame=single,
  numbers=none,
  showtabs=false,
  showspaces=false,
  showstringspaces=false,
  identifierstyle=\ttfamily,
  keywordstyle=\color{cBlue}
  commentstyle=\color{cGreen},
  stringstyle=\color{cRed},
  numberstyle=\color{black},
% \lstdefinestyle{C++}{language=C++,style=numbers}’.
}
\lstset{
  backgroundcolor=\color{lbcolor},
  tabsize=2,
  language=C++,
  captionpos=b,
  tabsize=3,
  frame=lines,
  numbers=none,
  numberstyle=\tiny,
  numbersep=5pt,
  breaklines=true,
  showstringspaces=false,
  basicstyle=\ttfamily,
  identifierstyle=\color{cOrange},
  keywordstyle=\color{cBlue},
  commentstyle=\color{cGreen},
  stringstyle=\color{cRed}
}

\lstdefinelanguage{makefile}{
  morekeywords={cc,CFLAGS,LFLAGS,OBJ,EXE},
  morecomment=[l]{\#}
}

\lstdefinestyle{makefile}{
  language=makefile,
  basicstyle=\ttfamily,
  keywordstyle=\color{cBlue},
  commentstyle=\color{cGreen},
  frame=lines,
  numbers=none,
  backgroundcolor=\color{lbcolor}
}

%header & footer
\pagestyle{fancy}
\lhead{Tibor Schneider}
\rhead{Seite \thepage}
\cfoot{\today} 

\renewcommand{\headrulewidth}{0.4pt}
\renewcommand{\footrulewidth}{0.4pt}
\usepackage[american]{circuitikz}
\ctikzset{resistor = european} 
\ctikzset{tripoles/mos style/arrows}
\ctikzset{voltage = european}

%Title of Document

\chead{Elektronik 2 - Formelsammlung} 

\begin{document}

%switch
\pgfcircdeclarebipole{}{\ctikzvalof{bipoles/interr/height 2}}{spst}{\ctikzvalof{bipoles/interr/height}}{\ctikzvalof{bipoles/interr/width}}{

    \pgfsetlinewidth{\pgfkeysvalueof{/tikz/circuitikz/bipoles/thickness}\pgfstartlinewidth}

    \pgfpathmoveto{\pgfpoint{\pgf@circ@res@left}{0}}
    \pgfpathlineto{\pgfpoint{.6\pgf@circ@res@right}{\pgf@circ@res@up}}
    \pgfusepath{draw}   
}
\def\pgf@circ@spst@path#1{\pgf@circ@bipole@path{spst}{#1}}
\tikzset{switch/.style = {\circuitikzbasekey, /tikz/to path=\pgf@circ@spst@path, l=#1}}
\tikzset{spst/.style = {switch = #1}}
\makeatother
\hfuzz=10pt

\begin{twocolumn}

\section{Analoge Wandler}

\subsection{Schrittweite, Auflösung}
\begin{tabular}{ll}
  \begin{tabular}{>{\begin{definition}}l<{\end{definition}} >{\begin{definition}}l<{\end{definition}}}
    $D$ & Digitaler Wert \\
    $B_{n-1}$ & Bitwert (0: LSB) \\
    $q$ & Quantisierungsschritt \\
    $K, G$ & Konstanten \\
    $A$ & Analoges Signal \\
  \end{tabular} &
  \begin{mtabular}{l}
    $\fmm D = \sum_{k = 0}^{n-1} B_k \cdot 2^k = \frac{V_{in} - V_{ref_n}}{V_{ref_p} - V_{ref_n}} \cdot 2^n$ \\
    $\fmm V_{out} = \frac{D}{2^n} (V_{ref_p} - V_{ref_n} + V_{ref_n})$ \\
    $\fmm q = \frac{V_{ref_p} - V_{ref_n}}{2^n}$
  \end{mtabular}
\end{tabular}

\subsection{Fehler}
\subsubsection{Offset-Fehler}
\begin{tabular}{ll}
  \begin{tabular}{l}
    
    \begin{tikzpicture}
      \begin{axis} [
        clip = false,
        width=0.45\columnwidth,
        height=0.4\columnwidth,
        axis lines=middle,
        xmin = 0, 
        xmax = 8,
        ymin = 0,
        ymax = 9.5,
        xlabel = {$D_{in}$},
        ylabel = {$A_{out}$},
        xtick={1,3,5,7},
        xticklabels={$001$, $011$, $101$, $111$},
        extra x ticks={0,2,4,6},
        extra x tick labels={$000$, $010$, $100$, $110$},
        extra tick style={major tick length = 0.7cm, yshift=-0.2cm},
      ]
        
        \addplot [draw=cBlue, mark=*, mark options={draw=cBlue, fill=cBlue}, dashed] coordinates {(0,0) (1,1) (2,2) (3,3) (4,4) (5,5) (6,6) (7,7) };
        \addplot [draw=cRed , mark=*, mark options={draw=cRed , fill=cRed }, dashed] coordinates {(0,1.1) (1,2.1) (2,3.1) (3,4.1) (4,5.1) (5,6.1) (6,7.1) (7,8.1) };
        
        \draw [densely dotted] (axis cs:0,1.1) -- (axis cs:-0.5,1.1);
        \draw [densely dotted] (axis cs:0,0) -- (axis cs:-0.5,0);
        \draw [<->] (axis cs:-0.5,0) -- node[left]{$E_{off}$} (axis cs:-0.5,1.1);
        
      \end{axis}
    \end{tikzpicture}
    
  \end{tabular}&
  \begin{tabular}{l}
    \begin{tikzpicture}
      \begin{axis} [
        clip = false,
        width=0.5\columnwidth,
        height=0.4\columnwidth,
        axis lines=middle,
        xmin = 0, 
        xmax = 9.5,
        ymin = 0,
        ymax = 8,
        xlabel = {$A_{in}$},
        ylabel = {$D_{out}$},
        xtick = {1,2,3,4,5,6,7,8},
        xticklabels={$1$,$$,$3$,$4$,$5$,$6$,$7$,$8$},
        ytick = {0,1,2,3,4,5,6,7},
        yticklabels = {$000$,$001$,$010$,$011$,$100$,$101$,$110$,$111$}
      ]
        
        \addplot [draw=cBlue, const plot, thick] coordinates {(0.5,0)(0.5,1)(1.5,2)(2.5,3)(3.5,4)(4.5,5)(5.5,6)(6.5,7)(9,7)};
        \addplot [draw=cRed , const plot, thick] coordinates {(1.8,0)(1.8,1)(2.8,2)(3.8,3)(4.8,4)(5.8,5)(6.8,6)(7.8,7)(9,7)};
        
        \draw [densely dotted] (axis cs:0.5,0) -- (axis cs:0.5,-1.2);
        \draw [densely dotted] (axis cs:1.8,0) -- (axis cs:1.8,-1.2);
        \draw [<->] (axis cs:0.5,-1.2) -- node[below]{$E_{off}$} (axis cs:1.8,-1.2);
                 
      \end{axis}
    \end{tikzpicture}
  \end{tabular}
\end{tabular}

\subsubsection{Verstärikungs- oder Gain-Fehler}

\begin{tabular}{ll}
  \begin{tabular}{l}
    \begin{tikzpicture}
      \begin{axis} [
        clip = false,
        width=0.5\columnwidth,
        height=0.4\columnwidth,
        axis lines=middle,
        xmin = 0, 
        xmax = 8,
        ymin = 0,
        ymax = 7.5,
        xlabel = {$D_{in}$},
        ylabel = {$A_{out}$},
        xtick={1,3,5,7},
        xticklabels={$001$, $011$, $101$, $111$},
        extra x ticks={0,2,4,6},
        extra x tick labels={$000$, $010$, $100$, $110$},
        extra tick style={major tick length = 0.7cm, yshift=-0.2cm},
      ]
       
        \addplot [draw=cBlue, mark=*, mark options={draw=cBlue, fill=cBlue}, dashed] coordinates {(0,0) (1,1) (2,2) (3,3) (4,4) (5,5) (6,6) (7,7) }; 
        \addplot [draw=cRed , mark=*, mark options={draw=cRed , fill=cRed }, dashed] coordinates {(0,0) (1,0.8) (2,1.6) (3,2.4) (4,3.2) (5,4) (6,4.8) (7,5.6) }; 
        
        \draw [densely dotted](axis cs:7,7) -- (axis cs:7.5,7);
        \draw [densely dotted](axis cs:7,5.6) -- (axis cs:7.5,5.6);
        \draw [<->] (axis cs:7.5,7) -- node[right]{$E_G$} (axis cs:7.5,5.6);
        
      \end{axis}
    \end{tikzpicture}
    
  \end{tabular}&
  \begin{tabular}{l}
    \begin{tikzpicture}
      \begin{axis} [
        clip = false,
        width=0.5\columnwidth,
        height=0.4\columnwidth,
        axis lines=middle,
        xmin = 0, 
        xmax = 9.5,
        ymin = 0,
        ymax = 8,
        xlabel = {$A_{in}$},
        ylabel = {$D_{out}$},
        xtick = {1,2,3,4,5,6,7,8},
        xticklabels={$1$,$$,$3$,$4$,$5$,$6$,$7$,$8$},
        ytick = {0,1,2,3,4,5,6,7},
        yticklabels = {$000$,$001$,$010$,$011$,$100$,$101$,$110$,$111$}
      ]
      
        \addplot [draw=cBlue, const plot, thick] coordinates {(0.5,0)(0.5,1)(1.5,2)(2.5,3)(3.5,4)(4.5,5)(5.5,6)(6.5,7)(9,7)};
        \addplot [draw=cRed , const plot, thick] coordinates {(0.4,0)(0.4,1)(1.2,2)(2,3)(2.8,4)(3.6,5)(4.4,6)(5.2,7)(9,7)};
        
        \addplot [draw=cBlue, dashed, mark=none] coordinates {(0,0)(7,7)};
        \addplot [draw=cRed , dashed, mark=none] coordinates {(0,0)(5.6,7)};
        
        \draw [densely dotted](axis cs:7,7) -- (axis cs:7,7.5);
        \draw [densely dotted](axis cs:5.6,7) -- (axis cs:5.6,7.5);
        \draw [<->] (axis cs:5.6,7.5) -- node[above]{$E_G$} (axis cs:7,7.5);
                         
      \end{axis}
    \end{tikzpicture}
  \end{tabular}
\end{tabular}

\subsection{Integrale Nichtlinearität INL}
Die Integrale Nichtlinearität bezeichnet die maximale Abweigung der Ausgangskurve von der idealen Gerade nach der Korrektur von Offset- und Gain-Fehler:


\begin{tabular}{ll}
  \begin{tabular}{l}
    \begin{tikzpicture}
      \begin{axis} [
        clip = false,
        width=0.5\columnwidth,
        height=0.4\columnwidth,
        axis lines=middle,
        xmin = 0, 
        xmax = 8,
        ymin = 0,
        ymax = 7.5,
        xlabel = {$D_{in}$},
        ylabel = {$A_{out}$},
        xtick={1,3,5,7},
        xticklabels={$001$, $011$, $101$, $111$},
        extra x ticks={0,2,4,6},
        extra x tick labels={$000$, $010$, $100$, $110$},
        extra tick style={major tick length = 0.7cm, yshift=-0.2cm},
      ]
       
        \addplot [draw=cBlue, mark=*, mark options={draw=cBlue, fill=cBlue}, dashed] coordinates {(0,0)(1,1)  (2,2)  (3,3)  (4,4)  (5,5)  (6,6)  (7,7)}; 
        \addplot [draw=cRed,  mark=*, mark options={draw=cRed,  fill=cRed }, dashed] coordinates {(0,0)(1,1.2)(2,2.4)(3,3.8)(4,4.6)(5,5.4)(6,6.2)(7,7)};
        
        \draw [densely dotted] (axis cs:3,3) -- (axis cs:5,3);
        \draw [densely dotted] (axis cs:3,3.8) -- (axis cs:5,3.8);
        \draw [>-<] (axis cs:5,2.82) -- node[right]{$E_{INL}$} (axis cs:5,3.98);
        
      \end{axis}
    \end{tikzpicture}
    
  \end{tabular}&
  \begin{tabular}{l}
    \begin{tikzpicture}
      \begin{axis} [
        clip = false,
        width=0.5\columnwidth,
        height=0.4\columnwidth,
        axis lines=middle,
        xmin = 0, 
        xmax = 9.5,
        ymin = 0,
        ymax = 8,
        xlabel = {$A_{in}$},
        ylabel = {$D_{out}$},
        xtick = {1,2,3,4,5,6,7,8},
        xticklabels={$1$,$$,$3$,$4$,$5$,$6$,$7$,$8$},
        ytick = {0,1,2,3,4,5,6,7},
        yticklabels = {$000$,$001$,$010$,$011$,$100$,$101$,$110$,$111$}
      ]
      
        \addplot [draw=cBlue, const plot, thick] coordinates {(0.5,0)(0.5,1)(1.5,2)(2.5,3)(3.5,4)(4.5,5)(5.5,6)(6.5,7)(9,7)};
        \addplot [draw=cRed , const plot, thick] coordinates {(0.4,0)(0.4,1)(1.2,2)(2.1,3)(2.9,4)(4.1,5)(5.2,6)(6.3,7)(9,7)};
        
        \addplot [draw=cBlue, dashed, mark=none] coordinates {(0,0)(7,7)};
        \addplot [draw=cRed , dashed, mark=none] coordinates {(0,0)(0.4,0.5)(1.2,1.5)(2.1,2.5)(2.9,3.5)(4.1,4.5)(5.2,5.5)(6.3,6.5)(7,7)};
        
        \draw [densely dotted] (axis cs:3.5,4) -- (axis cs:3.5,4.5);
        \draw [densely dotted] (axis cs:2.9,4) -- (axis cs:2.9,4.5);
        \draw [>-<] (axis cs:2.7,4.5) -- node[above]{$E_{INL}$} (axis cs:3.7,4.5);
                         
      \end{axis}
    \end{tikzpicture}
  \end{tabular}
\end{tabular}

\subsection{Differentielle Nichtlinearität DNL}
Die Differentielle Nichtlinearität bezeichnet den maximalen Fehler, der beim Inkrement des Datenwortes um ein LSB auftritt.
Die DNL ist die Differenz des Spannungsinkrements um ein LSB und dem idealen Spannungsinkrement $q$.

\begin{tabular}{ll}
  \begin{tabular}{l}
    \begin{tikzpicture}
      \begin{axis} [
        clip = false,
        width=0.5\columnwidth,
        height=0.4\columnwidth,
        axis lines=middle,
        xmin = 0, 
        xmax = 8,
        ymin = 0,
        ymax = 7.5,
        xlabel = {$D_{in}$},
        ylabel = {$A_{out}$},
        xtick={1,3,5,7},
        xticklabels={$001$, $011$, $101$, $111$},
        extra x ticks={0,2,4,6},
        extra x tick labels={$000$, $010$, $100$, $110$},
        extra tick style={major tick length = 0.7cm, yshift=-0.2cm},
      ]
       
        \addplot [draw=cBlue, mark=*, mark options={draw=cBlue, fill=cBlue}, dashed] coordinates {(0,0)(1,1)  (2,2)  (3,3)  (4,4)  (5,5)  (6,6)  (7,7)}; 
        \addplot [draw=cRed,  mark=*, mark options={draw=cRed,  fill=cRed }, dashed] coordinates {(0,0)(1,1.1)(2,2.6)(3,3.8)(4,4.6)(5,5.4)(6,6.2)(7,7)};
        
        \draw [densely dotted] (axis cs:1,1.1) -- (axis cs:4,1.1);
        \draw [densely dotted] (axis cs:2,2.6) -- (axis cs:3.5,2.6);
        \draw [densely dotted] (axis cs:3.5,2.1) -- (axis cs:4,2.1);
        \draw [<->] (axis cs:4,1.1) -- node[right]{$q$} (axis cs:4,2.1);
        \draw [>-<] (axis cs:3.5,2.8) -- node[above right, yshift=-2mm]{\footnotesize $E_{DNL}$} (axis cs:3.5,1.9);  
                     
      \end{axis}
    \end{tikzpicture}
    
  \end{tabular}&
  \begin{tabular}{l}
    \begin{tikzpicture}
      \begin{axis} [
        clip = false,
        width=0.5\columnwidth,
        height=0.4\columnwidth,
        axis lines=middle,
        xmin = 0, 
        xmax = 9.5,
        ymin = 0,
        ymax = 8,
        xlabel = {$A_{in}$},
        ylabel = {$D_{out}$},
        xtick = {1,2,3,4,5,6,7,8},
        xticklabels={$1$,$$,$3$,$4$,$5$,$6$,$7$,$8$},
        ytick = {0,1,2,3,4,5,6,7},
        yticklabels = {$000$,$001$,$010$,$011$,$100$,$101$,$110$,$111$}
      ]
      
        \addplot [draw=cBlue, const plot, thick] coordinates {(0.5,0)(0.5,1)(1.5,2)(2.5,3)(3.5,4)(4.5,5)(5.5,6)(6.5,7)(9,7)};
        \addplot [draw=cRed , const plot, thick] coordinates {(0.4,0)(0.4,1)(1.1,2)(2,3)(2.9,4)(4.1,5)(5.2,6)(6.3,7)(9,7)};
        
        \addplot [draw=cBlue, dashed, mark=none] coordinates {(0,0)(7,7)};
        \addplot [draw=cRed , dashed, mark=none] coordinates {(0,0)(0.4,0.5)(1.1,1.5)(2,2.5)(2.9,3.5)(4.1,4.5)(5.2,5.5)(6.3,6.5)(7,7)};
        
        \draw [densely dotted] (axis cs:1.1,2) -- (axis cs:1.1,4);
        \draw [densely dotted] (axis cs:0.4,0.5) -- (axis cs:0.4,4.5);
        \draw [densely dotted] (axis cs:1.4,4.5) -- (axis cs:1.4,4);
        \draw [<->] (axis cs:0.4,4.5) -- node[above]{$q$} (axis cs:1.4,4.5);
        \draw [>-<] (axis cs:0.9,4) -- node[below right, xshift=-2mm] {\footnotesize $E_{DNL}$} (axis cs:1.6,4);
                         
      \end{axis}
    \end{tikzpicture}
  \end{tabular}
\end{tabular}

Für monotone Wandler muss $E_{DNL} > -1 LSB$ betragen. sonst wird es ein nicht-monotoner Wandler.

\subsection{Quantisierung und Signal to Noise Ratio}
das Quantisierungsrauschen kann als lineare Rauschquelle modelliert werden. Pro bit erhäht sich das SNR um 6dB

$$P_Q = \frac{q^2}{12} \quad SNR \approx 1.76 + n \cdot 6 \text{dB}$$

\section{Digital-Analog Wandler DAC}

\subsection{Strom-DAC nach Parallelverfahren}

\begin{center}
\begin{circuitikz}[scale=0.6, transform shape]
  \draw (0,3) node[left]{\Large $V_{Ref}$} to [short, o-] (11,3);
  \draw (2,3) to [resistor, l_={\Large $R$}, *-] (2,1) to [switch] (2,-0.5) to [short, -o] (12,-0.5) node[right]{\Large $I_{Out}$};
  \draw (3.5,3) to [resistor, l_={\Large $R$}, *-] (3.5,1) to [switch, -*] (3.5,-0.5);
  \draw (5,3) to [resistor, l_={\Large $R$}, *-] (5,1) to [switch, -*] (5,-0.5);
  \draw (6.5,3) to [resistor, l_={\Large $R$}, *-] (6.5,1) to [switch, -*] (6.5,-0.5);
  \draw (8,3) to [resistor, l_={\Large $R$}, *-] (8,1) to [switch, -*] (8,-0.5);
  \draw (9.5,3) to [resistor, l_={\Large $R$}, *-] (9.5,1) to [switch, -*] (9.5,-0.5);
  \draw (11,3) to [resistor, l_={\Large $R$}, *-] (11,1) to [switch, -*] (11,-0.5);
  
  %decoder
  \draw [thick] (1,-0.5) to (1,1) to (-1.5, 1) to (-1.5,-0.5) to (1,-0.5);
  \draw (-0.25,0.25) node {\Large Decoder};
  \draw (-1.5,-0.125) to [short] (-2.5,-0.125);
  \draw (-1.5,0.25) to [short] (-2.5,0.25) node[left]{\Large $D$};
  \draw (-1.5,0.625) to [short] (-2.5,0.625);
  \draw [dotted] (1,0.25) to [short] (11,0.25);
\end{circuitikz}
\end{center}

Dieser Wandler hat eine garantierte Stetigkeit. Er benötigt jedoch $2^n$ Wiederstände und $2^n$ Schalter, sowie einen $n$-to-$2^n$ Decoder. 
Als Schalter eignen sich Mosfets sehr gut.

$$I_{Out} = D \cdot I = D \cdot \frac{V_{Ref}}{R}$$

\subsection{Voltage Scaling DAC}

\begin{center}
\begin{circuitikz}[scale=0.6, transform shape]
  \draw (0,2) node[left] {\Large $V_A$} to [short, o-*] (1,2) to [resistor, l_={\Large $R$}, -*] (3,2) to [resistor, l_={\Large $R$}, -*] (5,2)
    to [resistor, l_={\Large $R$}, -*] (7,2)  to [resistor, l_={\Large $R$}, -*] (9,2)  to [resistor, l_={\Large $R$}, -*] (11,2)
    to [resistor, l_={\Large $R$}, -*] (13,2)  to [resistor, l_={\Large $R$}, -*] (15,2) to [short, -o] (16,2) node[right]{\Large $V_B$};
  \draw (1,2) to [switch] (1,0) to (16,0) [short, -o] node[right]{\Large $V_{Out}$};
  \draw (3,2) to [switch, -*] (3,0);
  \draw (5,2) to [switch, -*] (5,0);
  \draw (7,2) to [switch, -*] (7,0);
  \draw (9,2) to [switch, -*] (9,0);
  \draw (11,2) to [switch, -*] (11,0);
  \draw (13,2) to [switch, -*] (13,0);
  \draw (15,2) to [switch, -*] (15,0);
  
  %decoder
  \draw [thick] (1,-0.5) to (1,-2) to (3.5, -2) to (3.5,-0.5) to (1,-0.5);
  \draw (2.25,-1.25) node {\Large Decoder};
  \draw (3.5,-1.625) to [short] (4.5,-1.625);
  \draw (3.5,-1.25) to [short] (4.5,-1.25) node[right]{\Large $D$};
  \draw (3.5,-0.825) to [short] (4.5,-0.825);
  \draw [dotted] (1,-1.25) to [short] (0,-1.25) to [short] (0,1) to [short] (15,1);
\end{circuitikz}
\end{center}

Bei diesem Wandler ist ein automatisierter Elektronik-Test möglich. der Wert $D$ muss im PROM gespeichert werden. 
$V_A$ und $V_B$ können variabel sein. 

\subsection{DAC nach Wägeverfahren}
\begin{center}
  \begin{circuitikz} [scale=0.6, transform shape]
    \draw (3,3) node[spdt, anchor=in, rotate=90] (B0) {};
    \draw (5,3) node[spdt, anchor=in, rotate=90] (B1) {};
    \draw (7,3) node[spdt, anchor=in, rotate=90] (B2) {};
    \draw (9,3) node[spdt, anchor=in, rotate=90] (B3) {};
    
    \draw (0,5) node[left]{\Large $V_{ref_p}$} to [short, o-] (8.69,5) to [short] (B3.out 1); 
    \draw (2.69,5) to [short, *-] (B0.out 1);
    \draw (4.69,5) to [short, *-] (B1.out 1);
    \draw (6.69,5) to [short, *-] (B2.out 1);
    
    \draw (11, 4.5) node[right]{\Large $V_{ref_n}$} to [short, o-] (3.31, 4.5) to [short] (B0.out 2);
    \draw (5.31,4.5) to [short, *-] (B1.out 2);
    \draw (7.31,4.5) to [short, *-] (B2.out 2);
    \draw (9.31,4.5) to [short, *-] (B3.out 2);
    
    \draw (1,5) to [short] (1,3) to [resistor, l_={\Large $8R$}] (1,1) to [short, -o] (11,1) node[right]{\Large $V_{Out}$};
    \draw (B0.in) node[above right, xshift=2mm]{\Large $B_0$} to [resistor, l_={\Large $8R$}, -*] (3,1);
    \draw (B1.in) node[above right, xshift=2mm]{\Large $B_1$} to [resistor, l_={\Large $4R$}, -*] (5,1);
    \draw (B2.in) node[above right, xshift=2mm]{\Large $B_2$} to [resistor, l_={\Large $2R$}, -*] (7,1);
    \draw (B3.in) node[above right, xshift=2mm]{\Large $B_3$} to [resistor, l_={\Large $R$ }, -*] (9,1);
  \end{circuitikz}
\end{center}

Dieser Wandler benötigt nur $n$ Schalter und $n+1$ wiederstände. 
Jedoch ist die Stetigkeit nicht garantiert.
Ausserdem sind grosse Wertebereiche für die Wiederstände notwendig.
Um den Ausgang $V_{Out}$ belasten zu können, muss ein Buffer dazugeschaltet werden.

$$V_{Out} = \frac{B_0 \cdot \frac{1}{8R} + B_1 \cdot \frac{1}{4R} + B_2 \cdot \frac{1}{2R} + B_3 \cdot \frac{1}{R}}{\frac{2}{R}}\cdot (V_{ref_p} - V_{ref_n}) + V_{ref_n}$$

\subsection{Kapazitiver DAC (Charge Scaling)}

\begin{center}
  \begin{circuitikz} [scale=0.6, transform shape]
    \draw (3,3) node[spdt, rotate=90] (B0) {};
    \draw (5,3) node[spdt, rotate=90] (B1) {};
    \draw (7,3) node[spdt, rotate=90] (B2) {};
    \draw (9,3) node[spdt, rotate=90] (B3) {};
   
    \draw (12,0.5) node[op amp] (opamp) {};    
    
    \draw (-1,5) node[left]{\Large $V_{ref_p}$} to [switch, o-](1,5) node[above, xshift=-1cm, yshift=2mm]{\Large $S_2$} to [short] (8.69,5) to [short] (B3.out 1); 
    \draw (2.69,5) to [short, *-] (B0.out 1);
    \draw (4.69,5) to [short, *-] (B1.out 1);
    \draw (6.69,5) to [short, *-] (B2.out 1);
    
    \draw (-1, 4.5) node[left]{\Large $V_{ref_n}$} to [switch, o-](1,4.5) node[below, xshift=-1cm]{\Large $S_0$} to [short] (9.31, 4.5) to [short] (B3.out 2);
    \draw (3.31,4.5) to [short, *-] (B0.out 2);
    \draw (5.31,4.5) to [short, *-] (B1.out 2);
    \draw (7.31,4.5) to [short, *-] (B2.out 2);
    
    \draw (1,5) to [short] (1,3) to [capacitor, l_={\Large $C$}] (1,1) to [short] (opamp.-);
    \draw (B0.in) node[above left, xshift=-2mm]{\Large $B_0$} to [capacitor, l_={\Large $C$}, -*] (3,1);
    \draw (B1.in) node[above left, xshift=-2mm]{\Large $B_1$} to [capacitor, l_={\Large $2C$}, -*] (5,1);
    \draw (B2.in) node[above left, xshift=-2mm]{\Large $B_2$} to [capacitor, l_={\Large $4C$}, -*] (7,1);
    \draw (B3.in) node[above left, xshift=-2mm]{\Large $B_3$} to [capacitor, l_={\Large $8C$ }, -*] (9,1);
  
    \draw (opamp.+) to [short] (10,0) to [short] (10,4.5) to [short, -*] (9.31,4.5);
    \draw (opamp.out) to [short] (13.2,2.5) to [capacitor, l={\Large $C_{fb}$}] (10.8,2.5) to [short] (opamp.-);
    \draw (13.2,2.5) to [short, *-] (13.2,3.5) to [switch] (10.8,3.5) node[above, xshift=1.2cm]{\Large $S_1$} to [short, -*] (10.8,2.5);
    \draw (opamp.out) to [short] (14,0.5) node[right] {\Large $V_{Out}$};
  
  \end{circuitikz}
\end{center}

Mit den zusätzlichen Schalter $S_0$ und $S_1$ werden die Anfangsbedingungen gesetzt. 
Mit dem Schalter $S_2$ wird der Ausgang gültig. 

$$C_1 = B_3 \cdot 8 C + B_2 \cdot 4C + B_1 \cdot 2C + B_0 \cdot C$$
$$C_2 = \overline{B_3} \cdot 8 C + \overline{B_2} \cdot 4C + \overline{B_1} \cdot 2C + \overline{B_0} \cdot C + C$$
$$V_{Out} = \frac{C_1}{C_1 + C_2} \cdot (V_{ref_p} - V_{ref_n}) + V_{ref_n} = \frac{C_1}{2^n \cdot C}  \cdot (V_{ref_p} - V_{ref_n}) + V_{ref_n}$$

\subsection{DAC mit R-2R-Netzwerk}

\begin{center}
  \begin{circuitikz}[scale=0.6, transform shape]
    \draw (1,2.5) node[spdt, rotate=270, yscale=-1] (B3) {};
    \draw (3,2.5) node[spdt, rotate=270, yscale=-1] (B2) {};
    \draw (5,2.5) node[spdt, rotate=270, yscale=-1] (B1) {};
    \draw (7,2.5) node[spdt, rotate=270, yscale=-1] (B0) {};
    
    \draw (11, 1) node[op amp] (opamp) {};
    
    \draw (0,5) node[left]{\Large $V_{ref_p}$} to [short] (1,5) to [resistor, l={\Large $R$}] (3,5) to [resistor, l={\Large $R$}] (5,5) to [resistor, l={\Large $R$}] (7,5) to [short] (9,5);
    \draw (1,5) to [resistor, l={\Large $2R$}, *-] (B3.in) node[right, yshift=-5mm] {\Large $B_3$};
    \draw (3,5) to [resistor, l={\Large $2R$}, *-] (B2.in) node[right, yshift=-5mm] {\Large $B_2$};
    \draw (5,5) to [resistor, l={\Large $2R$}, *-] (B1.in) node[right, yshift=-5mm] {\Large $B_1$};
    \draw (7,5) to [resistor, l={\Large $2R$}, *-] (B0.in) node[right, yshift=-5mm] {\Large $B_0$};
    
    \draw (B3.out 1) to [short](0.69, 1.5) to (opamp.-);
    \draw (B2.out 1) to [short, -*] (2.69,1.5);
    \draw (B1.out 1) to [short, -*] (4.69,1.5);
    \draw (B0.out 1) to [short, -*] (6.69,1.5);
    
    \draw (0, 0.5) node[left]{\Large $V_{ref_n}$} to [short] (opamp.+);
    \draw (B3.out 2) to [short, -*] (1.31,0.5);
    \draw (B2.out 2) to [short, -*] (3.31,0.5);
    \draw (B1.out 2) to [short, -*] (5.31,0.5);
    \draw (B0.out 2) to [short, -*] (7.31,0.5);
    \draw (9,5) to [resistor, l={\Large $2R$}] (9,3) to [short, -*] (9,0.5);
    
    \draw (opamp.out) to [short] (12.2,3) to [resistor, l={\Large $R$}] (9.8,3) to [short] (opamp.-);
    \draw (opamp.out) to [short] (13,1) node[right] {\Large $V_{Out}$};
    
  \end{circuitikz}
\end{center}

\subsection{Zahlverfahren (PWM)}

\begin{center}
  \begin{circuitikz}[scale=0.6, transform shape]
    \draw (2,1) node[spdt, xscale=-1] (sw) {};
    
    \draw (0,1.31) node[left]{\Large $V_{ref_p}$} to [short, o-] (sw.out 1);
    \draw (0,0.69) node[left]{\Large $V_{ref_n}$} to [short, o-] (sw.out 2);
    \draw (sw.in) to [resistor, l={\Large $R$}] (5,1) to [capacitor, l = {\Large $C$}, -o] (5,-1) node[below] {\Large $V_{A_{GND}}$};
    \draw (5,1) to [short, *-o] (6,1) node[right] {\Large $V_{Out}$};
    
    %mod N counter
    \draw [thick] (-6,2) to (-2,2) to (-2,0.5) to (-6,0.5) to (-6,2);
    \draw (-4,1.25) node {\Large mod $N$ Counter};
    \draw (-7,1.25) node[left]{\Large $N$} to [short] (-6,1.25);
    
    %<n?
    \draw [thick] (-6,-2) to (-2,-2) to (-2,-0.5) to (-6,-0.5) to (-6,-2);
    \draw (-4,-1.25) node {\Large $<$ $n$ ?};
    \draw (-7,-1.25) node[left]{\Large $n$} to [short] (-6,-1.25);
    \draw (-4,0.5) to (-4,-0.5);
    \draw [dotted] (-2,-1.25) to (2,-1.25) to (2,1);
  \end{circuitikz}
\end{center}

Der Schalter schaltet immer zwischen $V_{ref_p}$ und $V_{ref_n}$ um. 
der nachfolgende Tiefpass $RC$ glättet das ganze Signal. 
Diese Schaltung ist sehr einfach und ermöglicht eine hohe Auflösung.
Jedoch ist sie sehr langsam und benötigt eine grosse Zeitkonstante.
Es gilt folgende Beziehung: $\fmm V_{out} = V_{ref} \cdot \frac{n}{N}$

\section{Analog-Digital Wandler ADC}

\subsection{ADC nach Parallel-Verfahren}

\begin{tabular}{ll}

  \begin{tabular}{l}
    \begin{circuitikz}[scale=0.5, transform shape]
      \draw(0,18.25)
        node[left] {\Large $V_{in}$}
        to[short,o-](3,18.25)
        to[short](3,1.5);
    
      \draw(0, 17.5)
        node[left]{\Large $V_{ref}$}
        to [resistor, l_={\Large $\frac{3}{2}R$}, o-*](2, 17.5)
        to [resistor, l_={\Large $R$}, -*](2, 15)
        to [resistor, l_={\Large $R$}, -*](2, 12.5)
        to [resistor, l_={\Large $R$}, -*](2, 10)
        to [resistor, l_={\Large $R$}, -*](2, 7.5)
        to [resistor, l_={\Large $R$}, -*](2, 5)
        to [resistor, l_={\Large $R$}, -*](2, 2.5)
        to [resistor, l_={\Large $\frac{R}{2}$}](2, 0)
        node[ground]{};
    
      \draw(5, 17) %comparator
        node[op amp](opamp){}
        (opamp.-) to [short] (2,17.5)
        (opamp.+) to [short,-*] (3,16.5)
        (opamp.out) to [short] (7, 17);
    
      \draw(5, 14.5) %comparator
        node[op amp](opamp){}
        (opamp.-) to [short] (2,15)
        (opamp.+) to [short,-*] (3,14)
        (opamp.out) to [short] (7, 14.5);
    
      \draw(5, 12) %comparator
        node[op amp](opamp){}
        (opamp.-) to [short] (2,12.5)
        (opamp.+) to [short,-*] (3,11.5)
        (opamp.out) to [short] (7, 12);
    
      \draw(5, 9.5) %comparator
        node[op amp](opamp){}
        (opamp.-) to [short] (2,10)
        (opamp.+) to [short,-*] (3,9)
        (opamp.out) to [short] (7, 9.5);
    
      \draw(5, 7) %comparator
        node[op amp](opamp){}
        (opamp.-) to [short] (2,7.5)
        (opamp.+) to [short,-*] (3,6.5)
        (opamp.out) to [short] (7, 7);
    
      \draw(5, 4.5) %comparator
        node[op amp](opamp){}
        (opamp.-) to [short] (2,5)
        (opamp.+) to [short,-*] (3,4)
        (opamp.out) to [short] (7, 4.5);
    
      \draw(5, 2) %comparator
        node[op amp](opamp){}
        (opamp.-) to [short] (2,2.5)
        (opamp.+) to [short,-*] (3,1.5)
        (opamp.out) to [short] (7, 2);
    
      %component
      \draw [thick] (7, 18)
        to [short](10,18)
        to [short](10,0.5)
        to [short](7,0.5)
        to [short](7,18)
        (8.5, 8.7) node[label={\large Latch Encoder}]{};
    
      %clock
      \draw(5, -0.5)
        node[label={$CLK$}] {}
        to [short,o-](8.5, -0.5)
        to [short](8.5, 0.5);
    
      %Out
      \draw[thick](10, 9)
        to [short, -o](11, 9)
        node[label={\Large $Out$}]{};
    
      %Out Vector Declaration
      \draw(10.3, 9.3)
        node[label={\Large $3$}]{}
        to[short](10.7, 8.7);
    \end{circuitikz}
  \end{tabular} &
  \begin{tabular}{p{0.3\columnwidth}}
    Dieses Verfahren ist sehr schnell. es braucht jedoch $2^n$ Wiederstände und $2^n-1$ Komparatoren.
  \end{tabular}
\end{tabular}

\subsection{SAR: Successive Approximation Register}

\begin{center}
  \begin{circuitikz} [scale=0.6, transform shape]
    %comp
    \draw (5,2.5) node[op amp] (comp) {};
    
    %Search & Hold
    \draw (0,3) node[left] {\Large  $V_{in}$} to [short, o-] (1,3);
    \draw [thick] (1,3.75) to (3,3.75) to (3,2.25) to (1,2.25) to (1,3.75);
    \draw (2,3) node {\Large S\&H};
    \draw (3,3) to (comp.-);
    
    %n-bit SAR
    \draw (comp.out) to (7,2.5);
    \draw [thick] (7,3.25) to (10,3.25) to (10,1.75) to (7,1.75) to (7,3.25);
    \draw (8.5,2.5) node {\Large $N$-Bit SAR};
    \draw (6,4) node[left] {\Large $CLK$} to (8.5,4) to (8.5,3.25);
    
    %Register
    \draw (10,2.5) to (12,2.5);
    \draw (10.3,2.8) to (10.7,2.2) node[below left, yshift=1mm] {\Large $N$};
    \draw [thick] (12,3.25) to (15,3.25) to (15,1.75) to (12,1.75) to (12,3.25);
    \draw (13.5,2.5) node{\Large Register};
    \draw (15,2.5) to [short, -o] (16,2.5) node[right] {\Large $OUT$};
    
    %N-Bit DAC
    \draw (11,2.5) to [short, *-] (11,0.5) to (9, 0.5);
    \draw [thick] (9,1.25) to (6,1.25) to (6,-0.25) to (9,-0.25) to (9,1.25);
    \draw (7.5,0.5) node {\Large $N$-Bit DAC};
    \draw (6,0.5) to (3.5,0.5) to (3.5,2) to (comp.+);
    
  \end{circuitikz} 
\end{center}

Der Sample and Hold-Teil speichert das Eingangssignal während der Wandlung. 
der DAC-Ausgang nähert sich schrittweise approximativ dem Eingangssignal an.
Der Wandler startet in der Mitte des möglichen Bereichs und arbeitet sich dann immer weiter vor. 
Mit jedem Schritt wird die genauigkeit vergörssert. 
Als DAC sollte vorzugsweise ein kapazitiver DAC verwendet werden.

\begin{itemize}
  \item im \textbf{Sample Mode} werden alle Kondensatoren (vom DAC und SAR) mit $V_{in}$ geladen. 
  \item im \textbf{Hold Mode} wird der Eingang abgekoppelt. Danach bleibt die gesamte Ladung bestanden.
  \item im \textbf{Redistribution Mode} wird im DAC ein Schalter umgeschaltet. (im ersten Zyklus die MSB-Kap, $8C$). 
        Falls der Komparator-Ausgang $<0$ ist, so bleibt dieser Schalter und der nächste wird umgeschaltet.
        Falls $>0$ ist, wird der Schalter zurück-, und der nächste umgeschaltet.
\end{itemize}

\subsection{Dual Slope Wandler} 

\begin{tabular}{ll}
  \begin{tabular}{l}
    \begin{circuitikz}[scale=0.6, transform shape]
    
      \draw (5,0.5) node [op amp, yscale=1.02] (int) {};
      \draw (8,0)   node [op amp, yscale=1.02] (buf) {};
    
      \draw (0,2) node[left] {\Large $V_{ref}$} to [switch, o-] (2,2) node[above, yshift=2mm, xshift=-1cm]{\Large $S_2$} to [short, -*] (2,1);
      \draw (0,1) node[left] {\Large $V_{in}$}  to [switch, o-] (2,1) node[above, yshift=2mm, xshift=-1cm]{\Large $S_1$} to [resistor, l={\Large $R_i$}] (4,1) to [short] (int.-);
      \draw (int.+) to [short] (3.5,0) to [short] (3.5,-1) to [short, -o] (2.5,-1) node[left] {\Large $V_{A_{GND}}$};
      \draw (int.out) to [short] (buf.-);
      \draw (buf.+) to [short] (6,-0.5) to [short] (6,-1) to [short, -*] (3.5,-1); 
      \draw (6.5,0.5) node[below]{\Large $V_{int}$} to [short] (6.5,2) to [capacitor, l_={\Large $C_i$}] (3.8,2) to (3.8,1); 
      \draw (3.8,2) to [short, *-] (3.8,3.5) to [switch] (6.5,3.5) node[above, yshift=2mm, xshift=-1.3cm] {\Large $S_0$} to [short, -*] (6.5,2);
      \draw (buf.out) to [short, -o] ++(1,0) node[above]{\Large $CompOut$};
      
    \end{circuitikz}
  \end{tabular} &
  \begin{tabular}{l}
    \begin{tikzpicture}
      \begin{axis} [
        clip = false,
        width=0.4\columnwidth,
        height=0.3\columnwidth,
        axis lines=middle,
        xmin = 0, 
        xmax = 1,
        ymin = 0,
        ymax = 1,
        xlabel = {$t$},
        ylabel = {$V_{int}$},
        xtick = {0},
        ytick = {0}
      ]
        
        \draw [cRed, thick] (axis cs:0,0) -- (axis cs:0.5,0.7) -- (axis cs:0.8,0);
        \draw [clRed, thick] (axis cs:0,0) -- (axis cs:0.5,0.5) -- (axis cs:0.7143,0);
        \draw [clRed, thick] (axis cs:0,0) -- (axis cs:0.5,0.3) -- (axis cs:0.6286,0);
        \draw [dashed] (axis cs:0.5,0.7) -- (axis cs:0.5,0);
        \draw [>=latex, <->] (axis cs:0,-0.1) -- node[below]{$t_{int}$} (axis cs:0.5,-0.1);
        \draw [>=latex, <->] (axis cs:0.5,-0.1) -- node[below]{$t_{abint}$} (axis cs:0.8,-0.1);
      
      \end{axis}
    \end{tikzpicture}
  \end{tabular}
\end{tabular}

$$V_{int_{max}} = \int_0^{t_{int}} - \frac{V_{in}}{R_i \cdot C_i} \cdot dt = - \frac{V_{in}}{R_i \cdot C_i} \cdot t_{int}$$
%$$\text{Abintegration: } \quad V_{int}(t) = V_{int_{max}} - \frac{V_{ref}}{R_i \cdot C_i} \cdot t$$
$$t_{abint} = - \frac{V_{in} \cdot t_{int}}{V_{ref}} \Rightarrow V_{in} = - V_{ref} \cdot \frac{t_{abint}}{t_{int}}$$

Der Dual-Slope Wandler hat ein Tiefpass-verhalten. Ausserdem werden Frequenzen von 
$\fmm \frac{k}{t_{int} + t_{abint}}, k \in \mathbb{N}$ perfekt unterdrückt. Um den Netzbrumm (50Hz) zu Unterdrücken,
sollte die Integrationszeit 20ms betragen. 

\subsection{Sigma-Delta Wandler}

\begin{center}
  \begin{circuitikz}[scale=0.6, transform shape]
  
    \draw (5,0.5) node [op amp, yscale=1.02] (int) {};
    \draw (8,0)   node [op amp, yscale=1.02] (buf) {};
  
    \draw (0,3) node[left] {\Large $V_{ref}$} to [switch, o-] (2,3) node[above, yshift=2mm, xshift=-1cm]{\Large $S_3$} to [short, -*] (2,2);
    \draw (0,2) node[left] {\Large $V_{ref}$} to [switch, o-] (2,2) node[above, yshift=2mm, xshift=-1cm]{\Large $S_2$} to [resistor, l^={\Large $R_{i2}$}, -*] (3.8,2);
    \draw (0,1) node[left] {\Large $V_{in}$}  to [switch, o-] (2,1) node[above, yshift=2mm, xshift=-1cm]{\Large $S_1$} to [resistor, l_={\Large $R_{i1}$}] (3.8,1) to [short] (int.-);
    \draw (int.+) to [short] (3.5,0) to [short] (3.5,-1) to [short, -o] (2.5,-1) node[left] {\Large $V_{A_{GND}}$};
    \draw (int.out) to [short] (buf.-);
    \draw (buf.+) to [short] (6,-0.5) to [short] (6,-1) to [short, -*] (3.5,-1); 
    \draw (6.5,0.5) node[below]{\Large $V_{int}$} to [short] (6.5,2) to [capacitor, l_={\Large $C_i$}] (3.8,2) to (3.8,1); 
    \draw (buf.out) to [short] (11,0) node[above left]{\Large $CompOut$};
    
    %digital control
    \draw [thick] (11,-0.75) -- (11,0.75) -- (14,0.75) -- (14,-0.75) -- (11,-0.75);
    \draw (12.5,0) node {\Large Digital Ctrl};
    \draw (14,0) to [short, -o] (15,0) node[right]{\Large $D$};
    
  \end{circuitikz}
\end{center}

Sigma-Delta Wandler integireren wie Dual-Slope Wandler. Jedoch wird gleichzeitig auf- und abintegriert.
Statt Zähler kann ein digitales Filter (als Mittelwertbildung) verwendet werden.

\section{Filter}

\begin{tabular}{llll}
  \cdef{$f_{3dB}$} & \cdef{3dB-Grenze} \\
  \cdef{$Q$} & \cdef{Güte} \\
\end{tabular}

\subsection{Tiefpass-Filter 1. Ordnung}
\begin{center}
  \begin{tabular}{ll}
    \begin{tabular}{l}
      \begin{circuitikz} [scale=0.6, transform shape]
        \draw (0,2) node[left] {\Large $V_{in}$} to [resistor, l={\Large $R$}, o-*] (2,2) to [short, -o] (3,2) node[right] {\Large $V_{Out}$};
        \draw (2,2) to [capacitor, l={\Large $C$}] (2,0) node[ground]{}; 
      \end{circuitikz}
    \end{tabular} &
    \begin{tabular}{l}
      $\fmm G(s) = \frac{1}{1 + s \cdot RC}$ \\
      $\fmm T = R \cdot C$ \\
      $\fmm f_{3dB} = \frac{1}{2\pi \cdot RC}$ \\
    \end{tabular}
  \end{tabular}
\end{center}

\subsection{Filter 2. Ordnung}
\begin{tabular}{ll}
  \begin{tabular}{l}
    \begin{circuitikz} [scale=0.6, transform shape]
      \draw (0,2) node[left] {\Large $V_{in}$} to [resistor, l={\Large $R_1$}, o-*] (2,2) to [resistor, l={\Large $R_2$}, -*] (4,2) to [short, -o] (5,2) node[right] {\Large $V_{Out}$};
      \draw (2,2) to [capacitor, l={\Large $C1$}] (2,0) node[ground]{}; 
      \draw (4,2) to [capacitor, l={\Large $C2$}] (4,0) node[ground]{}; 
    \end{circuitikz}
  \end{tabular} &
  \begin{mtabular}{l}
    $ G(s) = \frac{1}{1 + s \cdot (C_1 R_1 + C_2 R_1 + C_2 R_2) + s^2 \cdot C_1 C_2 R_1 R_2}$ \\
    $\fmm Q = \frac{\sqrt{C_1 C_2 R_1 R_2}}{C_1 R_1 + C_2 R_1 + C_2 R_2}$ \\
  \end{mtabular}
\end{tabular}

\subsection{Sallen Key (Einfachmitkopplung)}

\begin{center}
  \begin{circuitikz}[scale=0.6, transform shape]
    \draw (6.5,3.5) node [op amp, yscale=-1.02](opamp){};
    
    \draw (0,4) node[left] {\Large $V_{in}$} to [resistor, l={\Large $R_1$}, o-*] (2,4) to [resistor, l={\Large $R_2$}, -*] (4,4) to (opamp.+);
    \draw (4,4) to [capacitor, l_={\Large $C_2$}] (4,2) node[ground] {};
    \draw (opamp.out) to [short, -*] (8,3.5) to [short, -o] (9,3.5) node[right] {\Large $V_{Out}$};
    \draw (2,4) to [short] (2,5.5) to [capacitor, l={\Large $C_1$}] (8,5.5) to [short] (8,3.5) to [resistor, l_={\Large $R_A$}, -*] (8,1.5) to [resistor, l={\Large $R_B$}] (10,1.5) node[ground]{};
    \draw (opamp.-) to [short] (5.3,1.5) to [short] (8,1.5);
  \end{circuitikz}
\end{center}

$$G(s) = \frac{\frac{R_A + R_B}{R_B}}{s^2 \cdot C_1 C_2 R_1 R_2 + s \cdot \left( C_2 \cdot (R_1 + R_2) + C_1 R_1 \cdot \left( 1-\frac{R_A + R_B}{R_B} \right) \right) + 1}$$
$$Q = \frac{\sqrt{C_1 C_2 R_1 R_2}}{C_2 \cdot (R_1 + R_2) + C_1 R_1 \cdot \left(1 - \frac{R_A + R_B}{R_B} \right)}$$

Diese Filter sind nicht geeignet für Systeme mit hohen Frequenzanteilen, da diese ebenfalls (leicht) durchgelassen werden. 
Es sind aber sehr grosse Polgüten möglich. 

\subsection{Multiple-Feedback Struktur}
\begin{center}
  \begin{circuitikz}[scale=0.6, transform shape]
    \draw (6,1.5) node [op amp, yscale=1.02] (opamp){};
    
    \draw (0,2) node[left] {\Large $V_{in}$} to [resistor, l_={\Large $R_1$}, o-*] (2,2) to [resistor, l={\Large $R_3$}, -*] (4,2) to [short] (opamp.-);
    \draw (2,2) to [capacitor, l={\Large $C_1$}] (2,0) node[ground]{};
    \draw (4,2) to [capacitor, l_={\Large $C_2$}, -*] (4,4);
    \draw (2,2) to [resistor,  l={\Large $R_2$}] (2,4) to [short] (7.5,4) to [short, -*] (7.5,1.5);
    \draw (opamp.out) to [short, -o] (8.5,1.5) node[right]{\Large $V_{Out}$};
    \draw (opamp.+) to [short] (4.5,1) to [short] (4.5,0) node[ground]{};
  \end{circuitikz}
\end{center}

$$G(s) = \frac{-\frac{R_2}{R_1}}{1 + s \cdot C_2 \cdot \left(R_2 + R_3 + \frac{R_2 R_3}{R_1}\right) + s^2 \cdot C_1 C_2 R_2 R_3}$$
$$Q = \frac{\sqrt{C_1 C_2 R_2 R_3}}{C_2 \cdot \left( R_2 + R_3 + \frac{R_2 R_3}{R_1} \right)}$$

\subsection{Maison-Regel}

$$ G(s) = \frac{\fmm \sum_k P_k \cdot \Delta_k}{\Delta}$$

\begin{tabular}{ll}
  $\Delta = $& $\fmm 1 - \sum \text{Schleife}$ \\
  & $\fmm + \sum \text{Produkt zweier nicht berührenden Schleifen}$ \\
  & $\fmm - \sum \text{Produkt dreier nicht berührenden Schleifen}$ \\  
  & $\fmm \pm \cdots$ \\
  $\Delta_k = $ & $\fmm 1-\sum \text{Schleife, die } P_k \text{ nicht berührt}$ \\
  & $\fmm +\sum \text{Prod. zweier sich und } P_k \text{ nicht berührenden Schleifen }$ \\
  & $\fmm -\sum \text{Prod. dreier sich und } P_k \text{ nicht berührenden Schleifen }$ \\
  & $\fmm \pm \cdots$ \\
\end{tabular}

\subsection{Ative Filter höherer Ordnung}



\begin{center}
  \begin{circuitikz} [scale=0.6, transform shape]
  
    %help lines
    %\draw [help lines, step=0.2,] (0,0) grid (16,5);
    %\draw [help lines, step=1, thick] (0,0) grid (16,5);
  
    \draw (3.5 ,0.5) node[op amp, yscale=1.02] (op1){};
    \draw (8.5 ,0.5) node[op amp, yscale=1.02] (op2){};
    \draw (13.5 ,0.5) node[op amp, yscale=1.02] (op3){}; 
    
    %opamp 1
    \draw (0,1) node[left] {\Large $V_{in}$}to [resistor, l={\Large $R_{in}$}, o-*] (2,1) to [short] (op1.-);
    \draw (op1.+) to [short] ++(-0.5,0) node[ground]{};
    \draw (op1.out) to [short] (5,0.5) to [short, -*] (5,1) to [short] (5,2.5) to [resistor, l={\Large $R_0$}] (2,2.5) to [short] (2,1);
  
    %opamp 2
    \draw (5,1) node[yshift=-0.9cm] {\Large $V_{opo1}$} to [resistor, l={\Large $R_{i1}$}, -*] (7,1) to [short] (op2.-);
    \draw (op2.+) to [short] ++(-0.5,0) node[ground]{};
    \draw (op2.out) to [short] (10, 0.5) to [short, -*] (10,1) to [short] (10,2.5) to [capacitor, l={\Large $C_{i1}$}] (7,2.5) to [short] (7,1);
    
    %opamp3
    \draw (10,1) node[yshift=-0.9cm] {\Large $V_{opo2}$} to [resistor, l={\Large $R_{i2}$}, -*] (12,1) to [short] (op3.-);
    \draw (op3.+) to [short] ++(-0.5,0) node[ground]{};
    \draw (op3.out) to [short] (15,0.5) to [short, *-] (15,1) to [short, -*] (15,2.5) to [capacitor, l={\Large $C_{i2}$}, -*] (12,2.5) to [short] (12,1);
    \draw (12,2.5) to [short] (12,3.5) to [resistor, l={\Large $R_{f2}$}] (15,3.5) to [short] (15,0.5) to [short, -o] (16,0.5) node[right] {\Large $V_{Out}$};
  
    %feedback
    \draw (2,2.5) to [short, *-] (2,5) to [resistor, l_={\Large $R_{fp}$}] (15,5) to [short, -*](15,2.5); 
  \end{circuitikz}
\end{center}

\begin{center}
  \begin{sfd}{0.8}
    %place a node:
    \node[terminal={left}{$V_{in} $}] (vin)  at (0,0) {};
    \node[terminal={above}{$I_{opn1}$}] (i1) at (2,0) {};
    \node[terminal={above}{$V_{opo1}$}] (v1) at (4,0) {};
    \node[terminal={above}{$I_{opn2}$}] (i2) at (6,0) {};
    \node[terminal={above}{$V_{opo2}$}] (v2) at (8,0) {};
    \node[terminal={above}{$I_{opn3}$}] (i3) at (10,0) {};
    \node[terminal={right}{$V_{Out}$}] (vout) at (12,0) {};
    
    %Draw connector
    \draw[amark=$\fmm \frac{1}{R_{in}}$] (vin) to (i1);
    \draw[amark=$\fmm -R_0$] (i1) to (v1);
    \draw[amark=$\fmm \frac{1}{R_{i1}}$] (v1) to (i2);
    \draw[amark=$\fmm \frac{-1}{s C_{i1}}$] (i2) to (v2);
    \draw[amark=$\fmm \frac{1}{R_{i2}}$] (v2) to (i3);
    \draw[amark=$\fmm \frac{-1}{s C_{i2}}$] (i3) to (vout);
    \draw[amarkback=$\fmm \frac{1}{R_{f2}}$] (vout) to [bend right=50] (i3);
    \draw[amarkback=$\fmm \frac{1}{R_{fp}}$, label revd] (vout) to [out=-100,in=-40] (i1);
  \end{sfd}
\end{center}

\section{Rauschen}
\begin{itemize}
  \item \textbf{Thermisches (weisses) Rauschen} ist über alle Frequenzen gleichverteilt. 
  \item \textbf{Flicker Noise} (rosa Rauschen). Es nimmt mit $\frac{1}{f}$ ab. 
    $\fmm E_n = K_v \cdot \sqrt{\ln{\frac{f_{max}}{f_{min}}}}$
  \item \textbf{Shot Noise} ist zufällig entstehender Peak. 
    $\fmm E_{sh} = k T \sqrt{\frac{2 B}{q I_{dc}}}$
  \item \textbf{Burst (popcorn) noise} sind diskrete hochfrequente Pulse (klingt wie Popcorn).
  \item \textbf{Avalanche Noise}: Entsteht in Dioden wenn Elektronen mit hoher Energie auf ein 
    Kristallgitter prallen (Lawineneffekt). 
\end{itemize}

\textbf{Bemerkungen zum Rauschen:} Es ist wichtig, die Rauschleistungen (und nicht die Spannungen) zu
addieren. Kapazitäten und Induktivitäten rauschen nicht. Da sie jedoch die Bandbreite verändern,
beeinflussen sie direkt die Rauschspannung.

\begin{center}
  \begin{tabular}{ll}
    \begin{dtabular}
      $v_n(t)$ & Rauschsignal \\
      $\overline{v_n}$ & Mittelwert \\
      $\overline{v_n^2}$ & Varianz \\
      $v_{n,rms}$ & Effektivwert \\
      $k$ & $ = 1.38 \cdot 10^{-23}$ Bolzmann-Konstante \\
      $T$ & Temperatur \\
      $T_t$ & Messdauer \\
      $B$ & Bandbreite \\
    \end{dtabular}
    \begin{mtabular}{l}
      $\fmm \overline{v_n} = \frac{1}{T_t} \int_{T_t} v_n(t) \cdot dt = 0$ \\
      $\fmm \overline{v_n^2} = \frac{1}{T_t} \int_{T_t} v_n^2(t) \cdot dt \neq 0$ \\
      $\fmm v_{n,rms} = \sqrt{\overline{v_n^2}}$  \\
    \end{mtabular}
  \end{tabular}
\end{center}

\subsection{Rauschleistungen}
\begin{center}
  \begin{mtabular}{ll}
    Weisses Rauschen & $\fmm \overline{v^2} = \int_{f_L}^{f_H} K \cdot df = K \cdot (f_H - f_L)$ \\
    $1/f$ Rauschen & $\fmm \overline{v^2} = \int_{f_L}^{f_H} \frac{K}{f} df = K \cdot \ln{\frac{f_H}{f_L}}$ \\
  \end{mtabular}
\end{center}

\subsection{Mittelwert von Rauschen}
\begin{center}
  \begin{tabular}{ll}
    \begin{tabular}{l}
      \begin{circuitikz}[scale=0.8, transform shape]
      
        \draw (0,3.5) to [R, l={\large $R$}, o-] (0,1.5) to [sV, l={\large $\overline{v_n^2}$}, -o] (0,0);
        \draw (2,3.5) to [short, o-*] (2,2.75) to [R, l_={\large $G$}, -*] (2,0.75) to [short, -o] (2,0)
              (2,2.75) to [short] (3,2.75) to [sI, l={\large $\overline{i_n^2}$}] (3,0.75) to [short] (2,0.75);
        \draw [<->] (3,1.4) to (3,2.1);
      \end{circuitikz}
    \end{tabular} &
    \begin{mtabular}{l}
      $\fmm \overline{v_n^2} = 4 k T B \cdot R$ \\
      $\fmm \overline{i_n^2} = 4 k T B \cdot G$ \\
      Serieschaltung: $\fmm \overline{v_{tot}^2} = \overline{v_1^2} + \overline{v_2^2}$
    \end{mtabular}
  \end{tabular}
\end{center}

\subsection{Rausch-Bandbreite} 

\begin{center}
  \begin{tabular}{ll}
    \begin{tabular}{l}
      \begin{circuitikz} [scale=0.8, transform shape]
        \draw (3,2) to [short, o-] (2,2) to [R, l_={\large $R$}] (0,2) to [sV, l_=\large $\overline{v_{in}^2}$] (0,0) to [short, -o] (3,0);
        \draw (2,2) to [C, l={\large $C$}, *-*] (2,0);
        \draw [->, shorten >=2mm, shorten <=2mm, bend left=40](3,2) to node[right] {\large $\overline{v_{out}^2}$} (3,0);
      \end{circuitikz}
    \end{tabular} & 
    \begin{mtabular}{l}
      $\fmm v_{out} = v_{in} \cdot \sqrt{ \int_0^\infty \frac{1}{1 + (2 \pi f \cdot R C)^2} \cdot df }$ \\
      $\fmm v_{out} = v_{in}\cdot \sqrt{\frac{1}{4 RC}}$ \\
      $\fmm \overline{v_{out}^2} = \overline{v_{in}^2} \cdot \frac{1}{4 RC} = \frac{k T}{C}$
    \end{mtabular}
  \end{tabular}
\end{center}

\subsection{Opamp-Modell mit Rauschen}

\begin{center}
  \begin{circuitikz}[scale=0.6,transform shape]

    %help lines
    %\draw [help lines, step=0.2, clGray] (-4,-1) grid (10,8);
    %\draw [help lines, step=1, thick, clGray] (-4,-1) grid (10,8);

    %connectors
    \node (op-) at (0,5) {};
    \node (op+) at (0,2) {};
    \node (opout) at (9,3.5) {};
    
    %big opamp
    \draw [thick] (1,7.5) to (8,3.5) to (1,-0.5) to (1,7.5);
    \draw (1,5) node[below left] {\large $-$};
    \draw (1,2) node[above left] {\large $+$};
    
    %small opamp
    \draw [thick] (3.5,5.286) to (7,3.5) to (3.5,1.714) to (3.5,5.286);
    \draw (3.5,5) node[below right] {\large $-$};
    \draw (3.5,2) node[above right] {\large $+$};
        
    \draw(op-) to [short] (3.5,5);
    \draw(op+) to [short] (2,2) to [sV, l_={\large $\overline{v_n^2}$}] (3.5,2);
    
    \draw (2,2) to [sI, l_={\large $\overline{i_{np}^2}$}, *-*] (2,3.5) to [sI, l_={\large $\overline{i_{nn}^2}$}, -*] (2,5);
    \draw [{Latex[scale=0.6]}-{Latex[scale=0.6]}] (2,2.4) to (2,3.1);
    \draw [{Latex[scale=0.6]}-{Latex[scale=0.6]}] (2,4.6) to (2,3.9);
    \draw (2,3.5) to [short] (1.25,3.5) node[ground, scale=0.6] {};
    \draw (7,3.5) to [short] (opout);
    
    \draw (op-) to [R, l={\Large $R_1$}] ++(-2,0) to [sV, l={\Large $\overline{v_1^2}$}] ++(-1.5,0) node[ground] {}; 
    \draw (op+) to [R, l={\Large $R_3$}] ++(-2,0) to [sV, l={\Large $\overline{v_3^2}$}] ++(-1.5,0) node[ground] {}; 
    \draw (op-) to [short, *-] (0,8) to [short] (3,8) to [sV, l={\Large $\overline{v_2^2}$}] (4.5,8) 
          to [R, l={\Large $\overline{R_2}$}] (6.5,8) to [short] (9,8) to [short, -*] (9,3.5) 
          to [short, -o] (10,3.5) node[right] {\Large $\overline{v_{tot}^2}$};
    
  \end{circuitikz}
\end{center}

Das Ausgangsrauschen $\overline{v_{out}^2}$ berechnet sich durch Superposition. 
Es werden immer alle ausser eine Rauschquelle ausgeschaltet und ihren Einfluss berechnet.
Danach werden alle Einflüsse zusammenaddiert. 

$$\overline{v_{tot}^2} = \int \left( 
    4kTR_2 A 
  + 4kTR_3 A^2
  + \overline{i_{nn}^2} R_2^2
  + \overline{I_{np}^2} R_3^2 A^2
  + \overline{v_n^2} A^2
\right) df$$

$$\text{mit} \quad A = \left(\frac{R_1 + R_2}{R_1}\right)$$

Bei CMOS-Opamps ist das stromrauschen meist zu vernachlässigen. Ausserdem kann der Wiederstand $R_3=0$
vernachlässigt werden. Ausserdem glit: \cdef{$e_w [Hz]$: Noise / Root (aus dem Datenblatt), $f_{enc} [Hz]$: 
Noise Corner Frequency (aus dem Datenblatt), $B_{en}$: Effective Noise Bandwidth}

$$\overline{v_{tot}^2} = B_{en} \cdot 4kTR_2 A + e_w^2 A^2 \left[ f_{enc} \ln{\left( \frac{f_H}{f_L} \right)} + B_{en} \right]$$

Wenn ausserdem die Bandbreite $B \gg f_{enc}$ (mindestens 10 mal grösser) ist, können weitere
Vernachlässigungen gemacht werden.

$$\overline{v_{tot}^2} = 4kT \cdot R_2 \cdot A \cdot B_{en} + e_w^2 \cdot A^2 \cdot B_{en} $$

\section{Spannungsreferenzen}

Die \cdef{Sensitivität $S$} ist die relative Ànderung der des Ausgangs zur Eingangsveränderung.
$\fmm S = \frac{\Delta V_{Ref} / V_{ref}}{\Delta V_{DD} / V_{DD}}$

\subsection{Einfachste Referenzquelle}
\begin{tabular}{ll}
  \begin{tabular}{l}
    \begin{circuitikz} [scale=0.8, transform shape]
      \draw (-0.5,4) node[left] {\large $V_{DD}$} to [short, o-] (0,4) to [R, l={\large $R_1$}]  (0,2) to [R, l={\large $R_2$}]  (0,0) node[ground] {};
      \draw (0,2) to [short, *-o] (1,2) node[right] {\large $V_{Ref}$};
    \end{circuitikz}
  \end{tabular} &
  \begin{tabular}{l}
    \begin{tikzpicture}
      \begin{axis} [
          clip = false,
          width=0.7\columnwidth,
          height=0.45\columnwidth,
          axis lines=middle,
          xmin = 0, 
          xmax = 1.2,
          ymin = 0,
          ymax = 1.2,
          xlabel = {$V$},
          ylabel = {$I$},
          xtick = {0.5,0.9,1,1.1},
          xticklabels={$V_{Ref}$, $V_{DD_{min}} \qquad \quad$, $V_{DD}$, $\qquad \quad V_{DD_{max}}$},
          ytick = {0},
          %yticklabels = {}
        ]
        
          \draw [thick] (axis cs:0,1) -- (axis cs:1,0); 
          \draw (axis cs:0,0.9) -- (axis cs:0.9,0);
          \draw (axis cs:0,1.1) -- (axis cs:1.1,0);
          
          \draw [thick] (axis cs:0,0) -- (axis cs:1.1,1.1);
          \draw [dashed, -*] (axis cs:0.5,0) -- (axis cs:0.5,0.525);
          
          \draw [dashed] (axis cs:0.9,0) -- (axis cs:0.9,0.4);
          \draw [dashed] (axis cs:1.1,0) -- (axis cs:1.1,0.4);
          \draw [<->] (axis cs:0.9,0.4) -- node[above] {$\Delta V_{DD}$} (axis cs:1.1,0.4);
          
          \draw [dashed] (axis cs:0.45,0.45) -- (axis cs:0.45,0.8);
          \draw [dashed] (axis cs:0.55,0.55) -- (axis cs:0.55,0.8);
          \draw [<->] (axis cs:0.45,0.8) -- node[above] {$\Delta V_{Ref}$} (axis cs:0.55,0.8);
      
      \end{axis}
    \end{tikzpicture}
  \end{tabular}
\end{tabular}

$$S = \frac{\Delta V_{Ref} / V_{ref}}{\Delta V_{DD} / V_{DD}} = 1$$

\subsection{Dioden Referenz}
\begin{tabular}{ll}
  \begin{tabular}{l}
    \begin{circuitikz} [scale=0.6, transform shape]
    
      %help lines
      %\draw [help lines, step=0.1, clGray] (0,0) grid (4,4);
      %\draw [help lines, step=0.5, thick, clGray] (0,0) grid (4,4);
    
      \draw (-0.5,4) node[left] {\Large $V_{DD}$} to [short, o-] (0,4) to [R, l={\Large $R_1$}] (0,2) to [D, l={\Large $D_2$}]  (0,0) node[ground] {};
      \draw (0,2) to [short, *-o] (0.5,2) node[right] {\Large $V_{Ref}$};
      
      \draw (3,1) node[pnp] (pnp) {};
      \draw (3,1) node[right] {\Large $Q_2$};
      \draw (2.5,4)  node[left] {\Large $V_{DD}$} to [short, o-] (3,4) to [R, l={\Large $R_1$}] (3,2) to [short] (pnp.emitter);
      \draw (pnp.collector) to (3,0) node[ground] {};
      \draw (pnp.base) to (2.15,0.24) to (pnp.collector);
      \draw (3,2) to [short, *-o] (3.5,2) node[right] {\Large $V_{Ref}$};
      
    \end{circuitikz}
  \end{tabular} &
  \begin{tabular}{l}
    \begin{tikzpicture}
      \begin{axis} [
          clip = false,
          width=0.65\columnwidth,
          height=0.45\columnwidth,
          axis lines=middle,
          xmin = 0, 
          xmax = 1.2,
          ymin = 0,
          ymax = 1.2,
          xlabel = {$V$},
          ylabel = {$I$},
          xtick = {0.616,1},
          xticklabels={$V_{Ref} \quad$, $V_{DD}$},
          ytick = {0},
          %yticklabels = {}
        ]
        
          \draw [thick] (axis cs:0,1) -- (axis cs:1,0); 
          \draw (axis cs:0,0.9) -- (axis cs:0.9,0);
          \draw (axis cs:0,1.1) -- (axis cs:1.1,0);
          
          \addplot [thick, domain=0:0.8, samples=100] coordinates 
          {
            (0.00000,0.00000)(0.01000,0.00000)(0.02000,0.00000)(0.03000,0.00000)(0.04000,0.00000)(0.05000,0.00000)(0.06000,0.00000)(0.07000,0.00000)(0.08000,0.00000)(0.09000,0.00000)(0.10000,0.00000)(0.11000,0.00000)(0.12000,0.00000)(0.13000,0.00000)(0.14000,0.00000)(0.15000,0.00000)(0.16000,0.00000)(0.17000,0.00000)(0.18000,0.00000)(0.19000,0.00000)(0.20000,0.00000)(0.21000,0.00000)(0.22000,0.00000)(0.23000,0.00000)(0.24000,0.00000)(0.25000,0.00001)(0.26000,0.00001)(0.27000,0.00001)(0.28000,0.00002)(0.29000,0.00002)(0.30000,0.00003)(0.31000,0.00004)(0.32000,0.00005)(0.33000,0.00007)(0.34000,0.00009)(0.35000,0.00013)(0.36000,0.00017)(0.37000,0.00023)(0.38000,0.00031)(0.39000,0.00042)(0.40000,0.00056)(0.41000,0.00076)(0.42000,0.00102)(0.43000,0.00138)(0.44000,0.00186)(0.45000,0.00251)(0.46000,0.00339)(0.47000,0.00458)(0.48000,0.00618)(0.49000,0.00835)(0.50000,0.01127)(0.51000,0.01521)(0.52000,0.02053)(0.53000,0.02772)(0.54000,0.03742)(0.55000,0.05051)(0.56000,0.06818)(0.57000,0.09203)(0.58000,0.12423)(0.59000,0.16769)(0.60000,0.22636)(0.61000,0.30555)(0.62000,0.41245)(0.63000,0.55675)(0.64000,0.75154)(0.65,0.93633)
          };
          \draw [dashed, -*] (axis cs:0.616,0) -- (axis cs:0.616,0.41);
          
          \draw [dashed] (axis cs:0.9,0) -- (axis cs:0.9,0.4);
          \draw [dashed] (axis cs:1.1,0) -- (axis cs:1.1,0.4);
          \draw [<->] (axis cs:0.9,0.4) -- node[above] {$\Delta V_{DD}$} (axis cs:1.1,0.4);
          
          %\draw [dashed] (axis cs:0.6,0.3) -- (axis cs:0.6,0.8);
          %\draw [dashed] (axis cs:0.55,0.55) -- (axis cs:0.55,0.8);
          %\draw [<->] (axis cs:0.45,0.8) -- node[above] {$\Delta V_{Ref}$} (axis cs:0.55,0.8);
      
      \end{axis}
    \end{tikzpicture}
  \end{tabular}
\end{tabular}

$$I_D =  I_S \cdot e^{\frac{V_{BE}}{m \cdot V T}} \approx I_S \cdot e^{\frac{q \cdot V_{BE}}{k T}} \quad \Rightarrow \quad V_{Ref} = \frac{k T}{q} \ln \frac{I_D}{I_s} \qquad S = \frac{1}{ln \frac{I}{I_S}} \ll 1$$

Diese Referenz-spannung kann nur im Bereich von $0.6 \text{ bis } 0.7 \text{ } V$ verwendet 
werden. Um dies zu umgehen, wird häufig eine Zenerdiode (in Gegenrichtung) verwendet. 
die häufigste Zener-Durchbruchspannung beträgt $5.6 \text{ } V$.

\subsection{Bootstrap Referenz (PTAT-Stromquelle)}

Diese Strom-Referenz Methode ist zwar unabhängig von $V_{DD}$, aber temperaturabhängig.
der \cdef{Temperaturkoeffizient $T_C$} gibt die diese an. 

\begin{tabular}{ll}
  \begin{tabular}{l}
    \begin{circuitikz} [scale=0.8, transform shape]
      \draw (1,5) node [pmos, xscale=-1] (M3) {};
      \draw (3,5) node [pmos] (M4) {};
      \draw (1,3) node [nmos, xscale=-1] (M1) {};
      \draw (3,3) node [nmos] (M2) {};
      \draw (1,1) node [pnp, xscale=-1] (Q1) {};
      \draw (5,5) node [pmos] (M5) {};
      
      %label transistors
      \draw (M1.B) node[left] {$M_1$};
      \draw (M2.B) node[right] {$M_2$};
      \draw (M3.B) node[left] {$M_3$};
      \draw (M4.B) node[right] {$M_4$};
      \draw (Q1)  node {$Q_1$};
      
      %connsct Transistors
      \draw (M3.S) to [short] (1,6);
      \draw (M4.S) to [short] (3,6);
      \draw (M5.S) to [short] (5,6);
      \draw (M3.G) to [short] (M4.G);
      \draw (M1.G) to [short] (M2.G);
      \draw (M3.D) to [short, i_={\large $I_1$}] (M1.D);
      \draw (M4.D) to [short, i={\large $I_2$}] (M2.D);
      \draw (M1.S) to [short] (Q1.E);
      \draw (Q1.C) to [short] (1,0);
      \draw (M2.S) to [R, l={$R$}] (3,0);
      \draw (Q1.B) to [short] (2,1) to [short, -*] (2,0) node[ground] {};
      \draw (M5.D) to [short, i={\large $I_5$}, -o] (5,3.5);
      
      %draw Base Connectors
      \draw (M1.D) to [short] ++(1,0)  to [short] (2,3);
      \draw (M4.D) to [short] ++(-1,0) to [short] (2,5);
      \draw (M4.D) to [short, *-] ++(1,0)  to [short] (4,5) to [short] (M5.B);
      
      %draw VDD and Ground
      \draw (0,6) node[left] {\large $V_{DD}$} to [short, o-] (5,6);
      \draw (1,0) to [short] (3,0);      
    \end{circuitikz}
  \end{tabular} &
  \begin{mtabular}{l}
    $I_1 = I_2$ \\
    $\fmm V_{Ref} = V_{EB1} = m V_T \cdot \ln \left( \frac{I_1}{I_s} \right)$ \\
    $\fmm V_{Ref} = V_{EB1} = I_2 \cdot R$ \\
    $\fmm V_T = \frac{kT}{q} \left[ \frac{V}{^\circ K} \right]$ \\
    $\fmm T_C = \frac{1}{V_t} \frac{dV_T}{dT} = \frac{1}{T}$ \\
  \end{mtabular}
\end{tabular}

\subsection{Bandgap-Referenzen}

\begin{center}
  \begin{circuitikz} [scale=0.6, transform shape]
    \draw (0,5) node[npn, xscale=-1] (npn) {};
    \draw (0,7.5) node[left] {\Large $V_{DD}$} to [I, o-] (0,6) to [short] (npn.C)
      (npn.E) to [short] (0,4) node[ground] {}
      (npn.C) to [short, *-] ++(1,0) to [short, -*] (1,5) to [short] (npn.B);
    
    %PTAT-Quelle
    \draw [thick] (-2,0) to (2,0) to (2,2) to (-2,2) to (-2,0)
      (0,1) node {\Large PTAT-Quelle};
      
    %gain
    \draw [thick] (4,2) to (6,1) to (4,0) to (4,2)
      (4.6,1) node {\Large $K$};
      
    %sum
    \draw [thick] (8,3) circle (0.6) node {\Large $\fmm \sum$};
    
    %connect
    \draw (2,1) to [short] (4,1)
      (6,1) to [short] (8,1) to [short] (8,2.4);
    \draw (1,5) to [short] (8,5) to [short] (8,3.6);
    \draw (8.6,3) to [short, -o] (10,3) node[right] {\Large $V_{Ref} = K \cdot V_T + V_{BE}$};
    
    %nodes
    \draw (1,5) node [above right] {\Large $V_{BE}$};
    \draw (3,1) node [above] {\Large $V_T$};
    \draw (7,1) node [above] {\Large $K \cdot V_T$};
    
    %Temperatur Koeffizient
    \draw [>=latex, <-] (4.5,5) to (4.5,6) node [above] {\Large $-2 \frac{mV}{^\circ K}$};
    \draw [>=latex, <-] (3,1) to (3,0) node [below] {\Large $0.085 \frac{mV}{^\circ K}$};
    \draw [>=latex, <-] (7,1) to (7,0) node [below] {\Large $2 \frac{mV}{^\circ K}$};
    \draw [>=latex, <-] (9.2,3) to (9.2,2) node [below] {\Large $0 \frac{mV}{^\circ K}$};
    
  \end{circuitikz}
\end{center}

Bei dieser Schaltung wird die Spannung $V_{BE}$ über einem pn-Übergang abgegriffen. Dieser
hat immer dieselbe negative Thermo-Spannung $V_T$. Nun wird eine PTAT-Quelle mit positivem 
Temperaturkoeffizienten verwendet, welche mit einer Verstärkung auf den (vom Betrag) gleichen
Wert bringt. Addiert man diese beiden Spannungen, so entsteht eine von der Temperatur 
unabhängige Referenzspannung.

\section{Spannungsquellen und -regler}

\subsection{Spannungsquelle mit Dioden}
\begin{tabular}{ll}
  \begin{tabular}{l}
    \begin{circuitikz}[scale=0.6, transform shape]
      \draw (0,4)
        to [V={\Large $V_Q$}] (0,0);
      \draw (0,4) 
        to [R, l={\Large $R$},-*] (2,4)
        to [\bipolLength=0.8cm, D, l={\Large $D_1$}] (2,3)
        to [\bipolLength=0.8cm, D, l={\Large $D_2$}] (2,2);
        
      \draw (2,1.5) node {\Large $\vdots$};
                        
      \draw (2,1)
        to [\bipolLength=0.8cm, D, l={\Large $D_n$}] (2,0)
        to [short] (0,0);
        
      \draw (2,4) 
        to [short, -o] (3,4) node [right] {\Large $V_{Out}$};
    \end{circuitikz}
  \end{tabular} & 
  \begin{mtabular}{l}
    $\fmm V_{Out} = n \cdot V_D \approx n \cdot 0.7V$ \\
    $\fmm i_{Out \, max} = \frac{V_{Q}-V_{Out}}{R}$ 
  \end{mtabular}
\end{tabular}

\subsection{Spannungsquelle mit Z-Diode}
\begin{tabular}{ll}
  \begin{tabular}{l}
    \begin{circuitikz}[scale=0.6, transform shape]
      \draw (0,2)
        to [V={\Large $V_Q$}] (0,0);
      \draw (0,2) 
        to [R, l={\Large $R$},-*] (2,2);
      \draw (0,0)
        to [short] (2,0)
        to [zDo] (2,2);
        
      \draw (2,2) 
        to [short, -o] (3,2) node [right] {\Large $V_{Out}$};
    \end{circuitikz}
  \end{tabular} &
  \begin{mtabular}{l}
    $\fmm V_{Out} = V_Z \approx 2.7V$ \\
    $\fmm i_{Out \, max} = \frac{V_{Q}-V_{Out}}{R}$ 
  \end{mtabular}
\end{tabular}


\subsection{Spannungsquelle mit Längstransistor}
\begin{tabular}{ll}
  \begin{tabular}{l}
    \begin{circuitikz} [scale=0.6, transform shape]
      \draw (2,4) node [npn, rotate=90] (npn) {};
      \draw (0,4) 
        to [short, o-*] (1,4)
        to [R, l_={\Large $R$}, -*] (1,2);
      \draw (0,0) 
        to [short, o-*] (1,0)
        to [zDo] (1,2);
      \draw (1,2) to [short] (2,2) to [short] (npn.B);
      \draw (1,4) to [short] (npn.C);
      \draw (3,4) to [short, o-] (npn.E);
      \draw (1,0) to [short, -o] (3,0);
      
      \draw [>=latex, ->, bend right=20, shorten <=2mm, shorten >=2mm] (0,4) to node[left] {\Large $V_{in}$} (0,0);
      \draw [>=latex, ->, bend left=20, shorten <=2mm, shorten >=2mm] (3,4) to node[right] {\Large $V_{out}$} (3,0);      
    \end{circuitikz}
  \end{tabular} &
  \begin{mtabular}{l}
    $\fmm V_{Out} = V_Z - V_{BE} \approx 2.7V - 0.7V$ \\
    $\fmm V_{Out \, max} = V_{in} - V_{BE}$
  \end{mtabular}
\end{tabular}
  
\subsection{Spannungsquelle mit Opamp}
\begin{tabular}{ll}
  \begin{tabular}{l}
    \begin{circuitikz} [scale=0.6, transform shape]
      \draw (2,5) node [npn, rotate=90] (npn) {};
      \draw (2,3) node [op amp, xscale=-1, rotate=90] (opamp) {};
      
      \draw (opamp.out) to [short] (npn.B);      
      \draw (0,5) to [short, o-] (npn.C);
      \draw (0,0) to [short, o-o] (5,0);
      \draw (opamp.+) to [V_={\Large $V_{ref}$}, -*] (1.5,0);
      \draw (opamp.-) to [short] ++(1.5,0) node (p1) {};
      \draw (4,5) to [R, l={\Large $R_1$}, *-*] (p1) to [R, l={\Large $R_2$}, -*] (4,0);
      \draw (npn.E) to [short, -o] (5,5);
      
      \draw [>=latex, ->, bend right=20, shorten <=2mm, shorten >=2mm] (0,5) to node[left] {\Large $V_{in}$} (0,0);
      \draw [>=latex, ->, bend left=20, shorten <=2mm, shorten >=2mm] (5,5) to node[right] {\Large $V_{out}$} (5,0);      
    \end{circuitikz}
  \end{tabular} &
  \begin{mtabular}{l}
    $\fmm V_{Out} = \frac{R_1 + R_2}{R_1} \cdot V_{ref}$
  \end{mtabular}
\end{tabular}

\subsection{Low-Drop-Out-Regler}
\begin{tabular}{ll}
  \begin{tabular}{l}
    \begin{circuitikz} [scale=0.6, transform shape]
      \draw (2,6) node [pnp, rotate=90] (pnp) {};
      \draw (3.5,3.5) node [op amp, rotate=180] (opamp) {};
      \draw (pnp.B) to [short] (2,3.5) to [short] (opamp.out);
      
      \draw (0,2) 
        to [V_={\Large $V_Q$}] (0,0);
      \draw (0,2) 
        to [short, *-] (0,6) 
        to [short] (pnp.E);
      \draw (0,2) 
        to [R, l_={\Large $R_1$}] (5,2) 
        to [short, *-] (5,3) 
        to [short] (opamp.-);
      \draw (pnp.C)
        to [short, -*] (6,6)
        to [R, l={\Large $R_3$}, -*] (6,4)
        to [R, l={\Large $R_2$}, -*] (6,0);
      \draw (5,0)
        to [zDo, *-] (5,2);
      \draw (opamp.+)
        to [short] (6,4);
      \draw (0,0)
        to [short, -o] (7,0);
      \draw (6,6)
        to [short, -o] (7,6);
          
      \draw [>=latex, ->, bend left=20, shorten <=2mm, shorten >=2mm] (7,6) to node[right] {\Large $V_{out}$} (7,0);      
    \end{circuitikz}
  \end{tabular} &
  \begin{tabular}{p{0.4\columnwidth}}
    Bei LDO-Regler darf der Spannungsabfall zwischen Ein- und Ausgang klein sein. 
    Da dies ein Regler ist, muss er eine ungerade Anzahl pfadinversionen haben (für Negative-Feedback).
  \end{tabular}
\end{tabular}
  
\subsection{Parallel- oder Shunt-Spannungsregler}
\begin{tabular}{ll}
  \begin{tabular}{l}
    \begin{circuitikz} [scale=0.6, transform shape]
      \draw (3,2) node [npn] (npn) {};
      
      \draw (0,4)
        to [V_={\Large $V_Q$}] (0,0);
      \draw (0,4) 
        to [R, l^={\Large $R_1$}, -*] (2,4)
        to [R, l_={\Large $R_2$}, -*] (2,2)
        to [R, l_={\Large $R_3$}, -*] (2,0);
      \draw (2,4)
        to [short, -*] (3,4)
        to [short] (npn.C);
      \draw (npn.E)
        to [short, -*] (3,0);
      \draw (npn.B)
        to [short] (2,2);
      \draw (3,4)
        to [short, -o] (4,4);
      \draw (0,0) 
        to [short, -o] (4,0);
      
      \draw [>=latex, ->, bend left=20, shorten <=2mm, shorten >=2mm] (4,4) to node[right] {\Large $V_{out}$} (4,0);      
    \end{circuitikz}
  \end{tabular} &
  \begin{mtabular}{l}
    $\fmm V_{Out} = V_{BE} \cdot \frac{R_2 + R_3}{R_2}$ \\
    $\fmm \frac{dV_{out}}{dT} = -2 \cdot \frac{R_2 + R_3}{R_2} \left[\frac{mV}{K}\right]$
  \end{mtabular}
\end{tabular}

\section{Ladungspumpen}

\subsection{Switched Capacitor Schaltung}
\begin{tabular}{ll}
  \begin{tabular}{l}
    \begin{circuitikz}[scale=0.6, transform shape]
      \draw (0,2) 
        to [switch, o-*] (2,2) node [above, xshift=-1cm, yshift=2mm] {\Large $S_1$}
        to [switch,  -*] (4,2) node [above, xshift=-1cm, yshift=2mm] {\Large $S_2$}
        to [short,   -o] (5,2);
      \draw (2,2)
        to [C, l={\Large $C_1$}, -*] (2,0);
      \draw (4,2)
        to [C, l={\Large $C_2$}, -*] (4,0);
      \draw (0,0)
        to [short, o-o] (5,0);
        
      \draw [>=latex, ->, bend right=30, shorten <=2mm, shorten >=2mm] (0,2) to node[left] {\Large $V_{in}$} (0,0);
      \draw [>=latex, ->, bend left=30, shorten <=2mm, shorten >=2mm] (5,2) to node[right] {\Large $V_{out}$} (5,0);      
   
    \end{circuitikz}
  \end{tabular} &
  \begin{tabular}{p{0.5\columnwidth}}
    Schalter $S_1$, $S_2$ sind nicht-überlappend. \\
    \textbf{Phase 1}: $S_1$ geschlossen, $C_1$ wird aufgeladen: $Q_1 = C_1 \cdot V_{in}$ \\
    \textbf{Phase 2}: $S_2$ geschlossen, Ladung $Q_1$ teilt sich auf $C_1$ und $C_2$ auf.
  \end{tabular}
\end{tabular}

Nach dem ersten Zyklus mit \cdef{$V_{out_0}$: Ausgangsspannung vor dem ersten Zyklus}:
$$Q_{tot_1} = C_1 \cdot V_{in} + C_1 \cdot V_{out_0} \qquad V_{out_1} = \frac{C_1 \cdot V_{in}}{C_1 + C_2}$$

Die Ausgangsspannung nähert sich der Eingangsspannung, wenn $C_1 = C_2$. 
Da sich die Spannung wie bei einer RC-Schaltung dem Endwert exponentiell annähert, kann ein Wiederstand $R_{eq}$ definiert werden (mit \cdef{$T$: Periodendauer}):
$$R_{eq} = \frac{T}{C}$$

\subsection{Spannungsinversion}

\begin{tabular}{ll}
  \begin{tabular}{l}
    \begin{circuitikz} [scale=0.6, transform shape]
      \draw (0,2) node [left] {\Large $V_{in}$}
        to [switch, o-*] (3,2) node [above, xshift=-1.5cm, yshift=2mm] {\Large $S_1$}
        to [switch] (6,2)  node [above, xshift=-1.5cm, yshift=2mm] {\Large $S_2$} node [ground] {};
      \draw (0,0) node [ground] {}
        to [switch, -*] (3,0) node [above, xshift=-1.5cm, yshift=2mm] {\Large $S_1$}
        to [switch, -o] (6,0) node [above, xshift=-1.5cm, yshift=2mm] {\Large $S_2$} node [right] {\Large $V_{out}$};
      \draw (3,2) node [above] {\Large $V_{CP}$}
        to [C, l={\Large $C$}] (3,0) node [below] {\Large $V_{CN}$};
    \end{circuitikz}
  \end{tabular} &
  \begin{tabular}{p{0.5\columnwidth}}
    \textbf{Phase 1}: Schalter $S_1$ geschlossen. $C$ wird aufgeladen. \\
    \textbf{Phase 2}: Schlater $S_2$ geschlossen. $V_{CP}$ springt auf $V_{GND}$. Also muss $V_{CN}$ auch denselben Sprung nach unten machen.
  \end{tabular}
\end{tabular}

$$V_{out} = -V_{in}$$

\subsection{Spannungsverdopplung}

\begin{tabular}{ll}
  \begin{tabular}{l}
    \begin{circuitikz} [scale=0.6, transform shape]
      \draw (0,2) node [left] {\Large $V_{in}$}
        to [switch, o-*] (3,2) node [above, xshift=-1.5cm, yshift=2mm] {\Large $S_1$}
        to [switch] (7,2)  node [above, xshift=-1.5cm, yshift=2mm] {\Large $S_2$}    
        to [C, l={\Large $C_2$}] (7,0) node [ground] {};
      \draw (7,2) 
        to [short, -o] (8,2) node [right] {\Large $V_{out}$};
      \draw (0,0) node [ground] {}
        to [switch, -*] (3,0) node [above, xshift=-1.5cm, yshift=2mm] {\Large $S_1$}
        to [switch, -o] (5,0) node [above, xshift=-1cm, yshift=2mm] {\Large $S_2$} node [right] {\Large $V_{in}$};
      \draw (3,2) node [above] {\Large $V_{CP}$}
        to [C, l_={\Large $C_1$}] (3,0) node [below] {\Large $V_{CN}$};
    \end{circuitikz}
  \end{tabular} &
  \begin{tabular}{p{0.4\columnwidth}}
    \textbf{Phase 1}: Schalter $S_1$ geschlossen. $C$ wird aufgeladen. \\
    \textbf{Phase 2}: Schlater $S_2$ geschlossen. $V_{CN}$ springt auf $V_{in} \: \rightarrow \: V_{CP}$ springt auf $2V_{in}$
  \end{tabular}
\end{tabular}


\section{Schaltregler}
% \begin{tabular}{ll}
%   \begin{dtabular}
%     $S$ & Stabilitätsfaktor \\
%     $R_a$ & Dyn. Ausgangswiederstand \\
%     $T_C$ & Temperaturkoeffizient \\
%   \end{dtabular} &
%   \begin{mtabular}{l}
%     $\fmm S = \frac{\Delta V_e / V_e}{\Delta V_a / V_a}$ \\
%     $\fmm R_a = \frac{dV_a}{dI_a}$ \\
%     $\fmm T_C = \frac{1}{V_a} \frac{dV_a}{dT}$ \\
%   \end{mtabular}
% \end{tabular}

\subsection{Aufwärtswandler}

\begin{ddtabular}
  $T_{on}$ & Dauer der Ladephase &
  $\Delta I_{L_{on}}$ & Stromdifferenz der Ladephase \\
  $T_{off}$ & Dauer der Entladephase &
  $\Delta I_{L_{off}}$ & Stromdifferenz der Entladephase \\
\end{ddtabular}

\begin{tabular}{ll}
  \begin{tabular}{l}
    \begin{circuitikz}[scale=0.6, transform shape]
      \draw (0,2) 
        to [L, l={\Large $L$},i={\Large $i_L$}, o-] (3,2) 
        to [\bipolLength=0.8cm, D, l={\Large $D$}, v>={\Large $V_D$}] (6,2)
        to [short, -o] (7,2);
      \draw (0,0) to [short, o-o] (7,0);
      \draw (3,2) to [switch, *-*] (3,0) node[left, yshift=1cm] {\Large $S$};
      \draw (6,2) to [C, l={\large $C$}, *-*] (6,0);

      \draw [>=latex, ->, bend right=20, shorten <=2mm, shorten >=2mm] (0,2) to node[left] {\Large $V_{in}$} (0,0);
      \draw [>=latex, ->, bend left=30, shorten <=2mm, shorten >=2mm] (7,2) to node[right] {\Large $V_{out}$} (7,0);
      
    \end{circuitikz} 
  \end{tabular} &
  \begin{tabular}{l}
    Lasephase: S geschlossen (on) \\
    Entladephase: S offen (off) \\
  \end{tabular}
\end{tabular}

$$\Delta I_{L_{on}}  = \frac{V_{in}}{L} \cdot T_{on} \qquad \Delta I_{L_{off}} = \frac{V_{in} - V_{out}}{L} \cdot T_{off}$$
$$\text{im Gleichgewicht:} \quad \Delta I_{L_{on}} = -\Delta I_{L_{off}} \qquad V_{out} = V_{in} \cdot \left(1 + \frac{T_{on}}{T_{off}}\right)$$

\subsection{Abwärtswandler}

\begin{tabular}{ll}
  \begin{tabular}{l}
    \begin{circuitikz} [scale=0.6, transform shape]
      \draw (0,3) to [switch, o-] (2,3) node [below, xshift=-1cm] {\Large $S$} 
        to [L, l={\Large $L$}, i={\Large $i_L$}] (5,3) to [short, -o] (6,3);
      \draw (0,0) to [short, o-o] (6,0);
      \draw (2,0) to [D, l={\Large $D$}, v>={\Large $V_D$}] (2,3);
      \draw (5,3) to [C, l={\Large $C$}] (5,0);
      
      \draw [>=latex, ->, bend right=20, shorten <=2mm, shorten >=2mm] (0,3) to node[left] {\Large $V_{in}$} (0,0);
      \draw [>=latex, ->, bend left=30, shorten <=2mm, shorten >=2mm] (6,3) to node[right] {\Large $V_{out}$} (6,0);
      
    \end{circuitikz}
  \end{tabular}
  \begin{tabular}{l}
    Lasephase: S geschlossen (on) \\
    Entladephase: S offen (off) \\
    $V_L = - \left(V_{out} + V_D\right)$
  \end{tabular}
\end{tabular}
$$\Delta I_{L_{on}} = \frac{V_{in} - V_{out}}{L} \cdot T_{on} \qquad \Delta I_{L_{off}} = \frac{V_{out} + V_D}{L} \cdot T_{off}$$
$$\text{im Gleichgewicht} \quad V_{out} = \frac{T_{on}}{T_{on} + T_{off}} \cdot V_{in} - \frac{T_{off}}{T_{on} + T_{off}} \cdot V_D \approx \frac{T_{on}}{T_{on} + T_{off}} \cdot V_{in} $$

Die Effizienz dieser Wandler kann verbessert werden, indem statt einer Silizium-Diode 
($V_D \approx 0.7V$) eine Schottky-Diode ($V_D \approx 0.3V$) verwendet wird. Noch besser
ist ein MOSFET, welcher nur einen On-Wiederstand $R_{DS_{on}}$ im $m\Omega$ bereich besitzt.

\subsection{Invertierender Wandler}

\begin{tabular}{ll}
  \begin{tabular}{l}
    \begin{circuitikz} [scale=0.6, transform shape]
      \draw (0,3) to [switch, o-] (3,3) node[below, xshift=-1.5cm] {\Large $S$} 
        to [switch] (6,3) [switch] node[below, xshift=-1.5cm] {\Large $\overline{S}$}
        to [short, -o] (7,3);
      \draw (0,0) to [short, o-o] (7,0);
      \draw (3,3) to [L, l_={\Large $L$}, v^>={\Large $V_L$}] (3,0);
      \draw (6,3) to [C, l={\Large $C$}] (6,0);
      
      \draw [>=latex, ->, bend right=20, shorten <=2mm, shorten >=2mm] (0,3) to node[left] {\Large $V_{in}$} (0,0);
      \draw [>=latex, ->, bend left=30, shorten <=2mm, shorten >=2mm] (7,3) to node[right] {\Large $V_{out}$} (7,0);
      
    \end{circuitikz}
  \end{tabular} &
  \begin{tabular}{l}
    $S = \text{on}$: erzeugung neg. Spannung \\
    $\fmm V_L = V_{in} = L \cdot \frac{\Delta I_L}{\Delta t}$ \\
    $S = \text{off}$: negative Spannung über $L$ \\
    $\fmm V_{out} = V_L = -L \cdot \frac{\Delta I_L}{\Delta t}$
  \end{tabular}
\end{tabular}

$$\text{wenn Eingeschwungen:} \quad V_{out} = V_L \cdot \frac{T_{on}}{T_{off}}$$

\section{Passive Elemente}
\subsection{Wiederstände}
\begin{tabular}{ll}
  \begin{dtabular}
    $\rho$ & Spez. Wiederstand \\
    $l$ & Länge des Leiters \\
    $A$ & Fläche des Leiters \\
  \end{dtabular} &
  \begin{mtabular}{c}
    $\fmm R = \rho \cdot \frac{l}{A}$ \\
    \begin{circuitikz} [scale=0.6, transform shape]
      \draw (0,1) to [L, l={\Large $L_s$}, o-*] (2,1) to [R, l={\Large $R$}, -*] (4,1) to [short, -o] (5,1);
      \draw (2,1) to [short] (2,0) to [C, l_={\Large $C_p$}] (4,0) to [short] (4,1);
    \end{circuitikz} \\
    $L_s \approx 5 \left[ nH \right] \qquad C_p \approx 0.5 \left[ pF \right]$
  \end{mtabular}
\end{tabular}

\subsection{Temperaturabhängige Wiederstände}

\subsubsection{PTC-Wiederstände}

Kaltleiter, positiver Temperaturkoeffizient: Wiederstand steigt mit steigernder \cdef{Temperatur $\theta \left[^\circ C\right]$}.

$$T > 0^\circ C: \quad R = R_0 \cdot \left( 1 + a \theta + b \theta^2 \right)$$
$$T < 0^\circ C: \quad R = R_0 \cdot \left(1 + a \theta + b \theta^2 + c \theta^3 \cdot (\theta - 100^\circ C) \right)$$

$$a = 3.9083 \left[ \frac{10^{-3}}{^\circ C} \right], \: b = -5.775 \left[\frac{10^{-7}}{^\circ C^2}\right], \: -4.183 \left[\frac{10^{-12}}{^\circ C^4}\right]$$

\subsubsection{NTC-Wiederstände} 

Heissleiter, negativer Temperaturkoeffizient: Wiederstand sinkt mit steigernder Temperatur:
\begin{center}
  \begin{ddtabular}
    $T [^\circ K]$ & Temperatur &
    $T_R [^\circ K]$ & Referenztemperatur \\
    $B$ & Materialkonstante &
    $R_R$ & Referenzwiederstand \\
  \end{ddtabular}
\end{center}

$$R = R_R \cdot e^{B \cdot \left( \frac{1}{T}- \frac{1}{T_R} \right)}$$

\subsection{Fotowiederstände LDR}

Wenn Licht auf die fotoempfindliche Fläche des LDR's trifft, verringert sich der Wiederstand. 
Dunkelwiederstand typischerweise zwischen $1 \ldots 100 M\Omega$, der Hellwiederstand (bei 1000 lx) typischerweise zwischen $0.1 \ldots 2 k\Omega$.
Die Wiederstandsänderung ist relativ langsam (mehrere $ms$ verzögert.)

\subsection{Kapazitäten}
\begin{tabular}{ll}
  \begin{dtabular}
    $A$ & Fläche des Kondensators \\
    $d$ & Abstand der Leiter \\
    $\varepsilon_0$ & $= 8.854 \cdot 10^{-12} \left[\frac{F}{m}\right]$  \\
    $\varepsilon_r$ & Relative Permitivität \\
    $R_{Leak}$ & Leckwiederstand \\
  \end{dtabular} &
  \begin{tabular}{c}
    $\fmm C = \frac{\varepsilon_r \cdot \varepsilon_0 \cdot A}{d}$\\
    \begin{circuitikz}[scale=0.6, transform shape]
      \draw (0,0) node [above] {\Large $+$} to [short, o-*] (1,0) to [C, l_={\Large $C$}, -*] (3,0) to [R, l={\Large $R_{e}$}] (5,0) to [L, l={\Large $L_{e}$}, -o] (7,0) node [above] {\Large $-$};
      \draw (1,0) to [short] (1,1) to [R, l={\Large $R_{Leak}$}] (3,1) to [short] (3,0); 
    \end{circuitikz}
  \end{tabular}
\end{tabular}

\subsection{Induktivitäten}
\begin{tabular}{ll}
  \begin{dtabular}
    $N$ & Windungszahl \\
    $\Phi$ & Magnetischer Fluss \\
    $I$ & Strom durch die Spule \\
    $\mu_0$ & $=4\pi \cdot 10^{-7} \left[\frac{N}{A^2}\right]$ \\
  \end{dtabular} &
  \begin{mtabular}{c}
    $\fmm L = \frac{N \cdot \Phi}{I}$\\
    $\fmm L_{Ring} = N^2 \frac{\mu_0 \mu_r A}{2 \pi r} \quad \fmm L_{Zylinder} = N^2 \frac{\mu_0 \mu_r A}{l}$ \\
    \begin{circuitikz}[scale=0.6, transform shape]
      \draw (0,1) to [short, o-*] (1,1) to [L, l_={\Large $L$}, -*] (3,1) to [R, l={\Large $R_{Cu}$}, -*] (5,1) to [short, -o] (6,1);
      \draw (1,1) to [short] (1,2) to [R, l={\Large $R_{Fe}$}] (3,2) to [short] (3,1);
      \draw (1,1) to [short] (1,0) to [C, l_={\Large $C_p$}] (5,0) to [short] (5,1);
    \end{circuitikz}
  \end{mtabular}
\end{tabular}

\end{twocolumn}
\end{document}
